\section{Introduction}

Multi-scalar multiplication is a bottleneck operation of several public and private cryptographic protocols, including elliptic curve digital signature algorithm (ECDSA) \cite{ECDSA}. Given positive integers $r_1, r_2$ and points on elliptic curves $P, Q$, we want to compute a point $S$ when 
$$S = r_1 P + r_2 Q.$$

There are several methods proposed to compute the operation. In most of those techniques, we pre-compute some elliptic points, and store them in our memory. In general, techniques with larger numbers of pre-computation points are usually faster than techniques with a smaller number. However, we cannot use those fast techniques in a computation environment with a limited amount of memory. 

By the number of pre-computation points, we can divide techniques proposed in literature into two types. The first type is techniques when the number of pre-computation points does not depend on the size of $r_1$ and $r_2$ such as Shamir’s trick \cite{Shamir}, interleaving method \cite{interleaving}, enlarged digit set \cite{enlarged2,enlarged4,enlarged1,enlarged3}, and double-base chain (DBC) \cite{dbc2,dbc3,dbc1}. The second type is techniques when the number of the points will be larger when $r_1$ and $r_2$ is larger such as addition chain \cite{additionChain1,additionChain2} and double-base number system (DBNS)  \cite{dbns}.

In this paper, we will focus on the multi-scalar multiplication based on DBNS. The technique for a scalar multiplication, an operation to compute $S = rP$ when $r$ is a positive integer and $P$ is an elliptic point, proposed by Meloni and Hasan in \cite{dbns}.  In the technique, the integer $r$ will be written in DBNS form $S_r \subseteq \mathbf{Z}_{\geq 0} \times \mathbf{Z}_{\geq 0}$ where
$$\sum_{(i,j) \in S_r} 2^I 3^j = r.$$ 
Since both $S^{(1)}_{41} = \{(2,2), (2,0), (0,0)\}$ and $S^{(2)}_{41} = \{(5,0), (0,2)\}$ are the DBNS form of 41, we know that the DBNS form of some $r$ is not unique.


Suppose that we have a DBNS form of $r$, $S_r$. Let $i_{\rm max}$ and $j_{\rm max}$ denote $\max\{ i' : (i',j') \in S_r \}$ and $\max\{ j' : (i',j') \in S_r \}$ respectively. The method calculates and stores the value of $3^iP$ for all $j \in \{j’ : (i’, j’) \in S_r \}$ using point triples. Then, for each $i \in \{i’ : (i’, j’) \in S_r \}$, it calculate $d(i)P$ where $d(i) := \sum\limits_{j \in \{j’ : (i,j’) \in S_r\}} 3^j $ by point additions. By those pre-computation, the point $rP$ can be obtained by
$$rP = 2\cdot \left( \dots \left( 2 \cdot \left( 2 \cdot d(i_{\rm max})P + d(i_{\mathrm{max} - 1}P) \right) + d(i_{\mathrm{max} - 2})P \right) \dots \right) + d(0)P.$$
For example, using $S_{41}^{(1)}$, we have $i_{\rm max} = j_{\rm max} = 2$, $d(2) = \left(3^2 + 3^0\right) = 10$, and $d(0) =  3^0 = 1$. The point $41P$ can be obtained by
$$41P = 2 \cdot \left( 2 \cdot d(2)P  \right) + d(0)P = 2 \cdot \left( 2 \cdot \left( 10P \right)  \right) + P.$$
The computation requires $j_{\rm max} - 1$ triples, $i_{\rm max} - 1$ doubles, and $|S_r| - 1$ additions. Since we have to store the points $d(i)P$ for all $i \in \{i : (i,j) \in S_r\}$, the number of the precomputation points can be as large as $|S_r|$ in the worst case.

In \cite{dbns2}, the DBNS form of a given integer $r$ is found using a greedy algorithm. Let $\mathcal{P}_r := \left\{ 2^i3^j : i,j \in \mathbf{Z} \text{ and } 2^i3^j \leq r \right\}$. When $S^{(a)}_r$ is the DBNS form found the algorithm, the set can be described as follows:
\[
S^{(a)}_r =
\begin{cases}
S^{(a)}_{r - 2^{i^*}3^{j^*}} \cup \{(i^*, j^*)\} & \text{if } 2^{i^*}3^{j*} := \max \mathcal{P}_r, \\
\emptyset       & \text{if } r = 0.
\end{cases}
\]
It is proved in the paper that, for any $r \in \mathbf{Z}_+$, we have $\left|S_r^{(a)}\right| \in O(\frac{\log r}{\log \log r})$. Later, by the result in \cite{dbns3}, we know that the upper bound is already tight, i.e. $\left|S_r^{(a)}\right| \in \Theta(\frac{\log r}{\log \log r})$.

By the discussion in the previous paragraphs, we need to store $\Theta(\frac{\log r}{\log \log r})$. That is why we categorize an algorithm in \cite{dbns} to the second type of the algorithms for the multi-scalar multiplication. In a computation environment with limited memory, we might not be able to store those $\Theta(\frac{\log r}{\log \log r})$ pre-computation points. Although a scalar multiplication method using DBNS is one of the fastest method, we cannot use the method in that situation.

\subsection{Our Contributions}

In this paper, we propose an algorithm that can reduce the number of pre-computation points of multi-scalar multiplication based on DBNS. Instead of storing all coefficients of $2^i$, $d(i)$, we choose to store all coefficients of $3^j$ in this paper. We show that the computation time of the multi-scalar multiplication can be significantly improved even when $j_{\rm max}$ is as small as $5$ or $6$. Since the number of pre-computation points is $2 \cdot j_{\rm max}$, setting $j_{\rm max}$ to $5$ or $6$ makes our number of pre-computation points equals those required in double-base chain \cite{experiment,dbc1}, interleaving method \cite{interleaving}, and Shamir's trick \cite{enlarged4}.

With the same number of pre-computation points, the experimental results in Section \ref{Section:Interleaving} shows that our algorithm is faster than any proposed method in literature when the size of $r$ is larger than $256$ bits. Our method is faster than the interleaving method by $???\%$, $???\%$, $???\%$, $???\%$, and $???\%$ when the number of bits is $256$, $320$, $384$, $448$, and $512$ respectively. Besides, the improvement over the tree-based double-base chain is are  $???\%$, $???\%$, $???\%$, $???\%$, and $???\%$ when the number of bits is $256$, $320$, $384$, $448$, and $512$.

Suppose that the value of $j_{\rm max}$ is fixed. For the situation when we have enough time and memory to find a suitable $S_r$, we propose a dynamic programming algorithm that can find a DBNS form of $r$, $S^*_r$, of which size is smaller than any other DBNS of $r$. As discussed before, the number of point additions required for the DBNS-based method is equal to the size of DBNS form. We can reduce the computation time by reducing the size of the form using our algorithm.

It is possible that $S^*_r$ is not a DBNS form that can optimize our computation time, since there might be a DBNS form $S'_r$ with a larger number of point additions but much smaller number of point doubles (much smaller $i_{\rm max}$). However, it can be seen that the different between the values of $i_{\rm max}$ of two different DBNS form of $r$ will not be more than $8$ or $9$, when $j_{\rm max}$ is equal to $5$ or $6$. By the fact that point doubles is not a time-consuming operation, we can say that the set $|S^*_r|$ can almost optimize the computation time of the DBNS-based multi-scalar multiplication.

We note that the time and memory consumption of our dynamic algorithm do not increase the cost of the multi-scalar multiplication. In most of cryptographic protocol, the scalars $r_1$ and $r_2$ are usually a private or public key. Generally, we perform several multi-scalar multiplications on embedded system using the same $r_1$ and $r_2$, after we generated those numbers on a personal computer. Therefore, it is reasonable to take time and memory to find the best DBNS form of $r_1$ and $r_2$ during their generation.

Motivated by the method in \cite{analysisMethod}, we propose an analysis method that can analyze the average number of point additions in the optimal DBNS-based multi-scalar multiplication. Given the same number of pre-compuation points, the analysis shows that our method requires much less point additions compared to the interleaving method with fractional windows \cite{fractional} or the enlarged digit set \cite{analysisMethod}.

One might think that our algorithm is merely a double-and-add technique with a digit set $\left\{0, \pm 3^0, \pm 3^1, \dots, \pm 3^{j_{\rm max}} \right\}$. However, our method is different from that technique. In the technique, only one point addition will be performed between two sets of point doubles, while more than one point additions are allowed to perform between those two sets.
