\section{Introduction}

Multi-scalar multiplication is a bottleneck operation of several public- and symmetric-key cryptographic protocols,
including elliptic curve digital signature algorithm (ECDSA) \cite{ECDSA}.
Given positive integers $m, n$ and points on elliptic curves $P_1, P_2$,
we want to compute a point $R$ where
$$R = m P_1 + n P_2.$$

There are several proposed methods to compute such operation.
Most of those techniques are to precompute some elliptic points  and store them in a memory.
In general, techniques with a larger number of precomputation points are usually faster than techniques with a smaller number of precomputation points.
However, those fast techniques cannot be used in the environment where only limited amount of memory is allowed.

By the number of precomputation points, we can categorize techniques proposed in the literature into two types.
The first type is techniques where the number of precomputation points does not depend on the size of scalars $m$ and $n$ such as Shamir’s trick \cite{Shamir},
interleaving method \cite{interleaving}, enlarged digit set \cite{enlarged2,enlarged4,enlarged1,enlarged3}, and double-base chain (DBC) \cite{dbc2,dbc3,dbc1}.
The second type is techniques where the number of the points grows larger when scalars $m$ and $n$ are getting larger such as addition chain \cite{additionChain1,additionChain2}
and double-base number system (DBNS)  \cite{dbns}.

In this paper, we focus on multi-scalar multiplication using a technique which belongs to the second type.
To be more precisely, we extended the scalar multiplication, i.e., an operation to compute $R = nP$ where $n$ is a positive integer and $P$ is an elliptic point, 
based on DBNS proposed by M\'eloni and Hasan in \cite{dbns}.
In their technique, an integer $n$ is written in a DBNS expansion $S_n \subseteq \mathbf{Z}_{\geq 0} \times \mathbf{Z}_{\geq 0}$ where
$$\sum_{(i,j) \in S_n} 2^i 3^j = n.$$ 
Since both $S^{(1)}_{41} = \{(2,2), (2,0), (0,0)\}$ and $S^{(2)}_{41} = \{(5,0), (0,2)\}$ are DBNS expansions of $41$, we know that a DBNS expansion of some $n$ is not unique.

Suppose $S_n$ is a DBNS expansion of $n$. Let $i_{\rm max}$ and $j_{\rm max}$ denote $\max\{ i' : (i',j') \in S_n \}$ and $\max\{ j' : (i',j') \in S_n \}$ respectively.
The method \cite{dbns} calculates and stores values $2^iP$ for all $i \in \{i : (i, j) \in S_n \}$ using point doubling.
Then, for each $j \in \{j : (i, j) \in S_n \}$, it calculates $d(j)P$ where $d(j) := \sum\limits_{i \in \{i’ : (i',j’) \in S_n\}} 2^i $ by point addition.
Therefore, $nP$ can be obtained by
$$nP = 3\cdot \left( \dots \left( 3 \cdot \left( 3 \cdot d(i_{\rm max})P + d(i_{\mathrm{max} - 1}P) \right) + d(i_{\mathrm{max} - 2})P \right) \dots \right) + d(0)P.$$
Using $S_{41}^{(1)}$ as an example, considering $i$ then points $2^0P$ and $2^2P$ are precomputed.
Considering $j$ then $d(0) = 2^2P + 2^0P = 5P$ for $(2,0),(0,0)$ and $d(2) = 2^2P = 4P$ for $(2,2)$ are also precomputed.
Therefore, $S_{41}^{(1)}$ can be obtained by
$$S_{41}^{(1)} = 3 \cdot \left( 3 \cdot d(2)P  \right) + d(0)P = 3 \cdot \left( 3 \cdot \left( 4P \right)  \right) + 5P = 41P.$$
%The computation requires $j_{\rm max} - 1$ triples, $i_{\rm max} - 1$ doubles, and $|S_n| - 1$ additions. --> No!!!
%precomp: i_{max}-1 doublings and in worst case i_{max} additions
%main comp: j_{max}-1 triplings and $|S_n} - 1$ additions
Since we have to store the points $2^iP$ for all $i \in \{i : (i,j) \in S_n\}$, the number of the precomputation points can be as large as $|S_n|$ in the worst case.

In \cite{dbns2}, a DBNS expansion of a given integer $n$ is obtained using a greedy algorithm.
Let $\mathcal{P}_n := \left\{ 2^i3^j : i,j \in \mathbf{Z} \text{ and } 2^i3^j \leq n \right\}$.
If $S^{(a)}_n$ is a DBNS expansion obtained by this algorithm,
it can be described as follows:
\[
S^{(a)}_n =
\begin{cases}
S^{(a)}_{n - 2^{i^*}3^{j^*}} \cup \{(i^*, j^*)\} & \text{if } 2^{i^*}3^{j*} := \max \mathcal{P}_n, \\
\emptyset       & \text{if } n = 0.
\end{cases}
\]
It is proved in the paper that, for any $n \in \mathbf{Z}_+$, we have $\left|S_n^{(a)}\right| \in O(\frac{\log n}{\log \log n})$.
Later, by the result in \cite{dbns3}, we know that the upper bound is already tight, i.e. $\left|S_n^{(a)}\right| \in \Theta(\frac{\log n}{\log \log n})$.

By the discussion in previous paragraphs, we need to store $\Theta(\frac{\log r}{\log \log r})$ precomputation points.
That is why we categorize the algorithm in \cite{dbns} to the second type of algorithms for the scalar multiplication.
In a computational environment with limited memory, we might not be able to store those $\Theta(\frac{\log r}{\log \log r})$ precomputation points.
Although a scalar multiplication method using DBNS is one of the fastest methods, we cannot use this method in such situation.

\subsection{Our Contributions}

In this paper, we propose an algorithm for multi-scalar multiplication that based on DBNS representation.
Compared to the DBNS-base scalar multiplication in \cite{dbns},
we significantly reduce the number of precomputation points that need to be stored.
Instead of storing all coefficients of $2^i$ and $d(j)$, we choose to store all coefficients of $3^j$.
We show that the computational time of the multi-scalar multiplication can be improved even when $j_{\rm max}$ is as small as $5$ or $6$.
Since the number of precomputation points is $2 \cdot j_{\rm max}$,
setting $j_{\rm max}$ to $5$ or $6$ makes our number of precomputation points approximately the same as those required in
double-base chain \cite{experiment,dbc1}, interleaving method \cite{interleaving}, and Shamir's trick \cite{enlarged4}.

With the same number of precomputation points, the experimental results in Section \ref{sec:interleaving}
shows that our algorithm is faster than any proposed method in literature when the size of $r$ is larger than $256$ bits.
Our method is faster than the interleaving method by $0.56\%$, $1.13\%$, $1.35\%$, $1.53\%$, and $1.49\%$
when the number of bits is $256$, $320$, $384$, $448$, and $512$ respectively.
Besides, the improvement over the tree-based double-base chain is are 
$4.41\%$, $4.74\%$, $4.89\%$, $5.00\%$, $5.14\%$, and $4.87\%$ when the number of bits is $192$, $256$, $320$, $384$, $448$, and $512$.

Suppose that the value of $j_{\rm max}$ is fixed.
For situations where we have enough time and memory to find a suitable $S_n$,
we propose a dynamic programming algorithm that finds a DBNS expansion $S^*_n$ of $n$, of which the size is smaller than any other DBNS expansions of $n$.
As discussed before, the number of point additions required for DBNS-based methods is equal to the size of DBNS expansion.
Our algorithm reduces the computational time by reducing the size of the expansion.

% Don't quite understand the message that this paragraph tries to convey.
It is possible that $S^*_n$ is not a DBNS expansion that can optimize the computational time,
since there might be a DBNS expansion $S'_r$ with a larger number of point additions but much smaller number of point doublings (much smaller $i_{\rm max}$).
However, it can be seen that the different between values $i_{\rm max}$ of two different DBNS expansions of $n$
will not be more than $8$ or $9$ when $j_{\rm max}$ is equal to $5$ or $6$.
By the fact that point doubling is not a time-consuming operation,
we can say that the set $|S^*_n|$ can almost optimize the computation time of the DBNS-based multi-scalar multiplication.

In most cryptographic protocols, scalars $m$ and $n$ are usually private- or public-keys.
Generally, we perform several multi-scalar multiplications on embedded system using the same $m$ and $m$ after those numbers are generated on a personal computer.
Therefore, it is reasonable to spend time and memory to find the best DBNS expansion of $m$ and $n$ during the generation.

Motivated by the method in \cite{analysisMethod},
we propose an analysis that analyzes the average number of point additions in the optimal DBNS-based multi-scalar multiplication.
Given the same number of precomputation points, the analysis shows that our method requires much less point additions compared to
the interleaving method with fractional windows \cite{fractional} and the enlarged digit set \cite{analysisMethod}.

One might think that our algorithm is merely a double-and-add with a digit set $\{0,$ $\pm 3^0,$ $\pm 3^1,$ $\dots,$ $\pm 3^{j_{\rm max}}$ $\}$.
However, these two algorithms are not exactly the same.
Our algorithm allows more than one point additions to be performed between two doublings.
while the double-and-add allows only one point addition to be performed.
