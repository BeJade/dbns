\documentclass[a4paper,11pt,runningheads]{llncs}

%\usepackage[margin=2.7cm]{geometry}

\pdfpagewidth=210 true mm
\pdfpageheight=297 true mm

\usepackage[table]{xcolor}
\definecolor{linkcolor}{rgb}{0.65,0,0}
\definecolor{citecolor}{rgb}{0,0.65,0}
\definecolor{urlcolor}{rgb}{0,0,0.65}
\usepackage[breaklinks=true,colorlinks=true, backref=page, linkcolor=linkcolor, urlcolor=urlcolor, citecolor=citecolor]{hyperref}
\usepackage{hyperref}

\usepackage{amsfonts,amsmath,amscd,amssymb,array}
\usepackage{algorithm, algpseudocode}
\usepackage{tabularx,booktabs}
\usepackage{array}
\usepackage{multirow}
\usepackage{float}
\floatname{listing}{Listing}

%\def\subheading#1{\medskip\refstepcounter{subsection}\noindent{\boldmath\textbf{\thesubsection. #1.} }\ignorespaces}	% with number ex. 1.1 Subheading
\def\subheading#1{\medskip\noindent{\boldmath\textbf{#1}}~\ignorespaces}	% without number ex. Subheading

\def\mul{\text{\bf M}}
\def\sqr{\text{\bf S}}
\def\con{\text{\bf m}}


%\def\setof#1{\mathord{\left\lbrace{#1}\right\rbrace}}
%\def\floor#1{\mathord{\left\lfloor{#1}\right\rfloor}}
%\def\min#1{\operatorname{min}\mathopen{}\setof{#1}}
%\def\max#1{\operatorname{max}\mathopen{}\setof{#1}}
%\def\Z{\mathord{\text{\bf Z}}}
%\def\F{\mathord{\text{\bf F}}}
%\def\F{\mathbb{F}}

%\numberwithin{table}{section}
%\numberwithin{figure}{section}
%\numberwithin{theorem}{section}

%\def\dash{$\,$---$\,$}

%\makeatletter
%\renewcommand\labelitemi{$\m@th\bullet$}
%\let\c@figure\c@subsection
%\let\c@table\c@subsection
%\makeatother

%\spnewtheorem{mytheorem}[subsection]{Theorem}{\bfseries}{\itshape}




\begin{document}

\titlerunning{Multi-Scalar Multiplication Using DBNS }
\title{Fast Elliptic Curve Multi-Scalar Multiplication Using DBNS with Constant Number of Pre-Computation Points}
\authorrunning{C. Chuengsatiansup, V. Suppakitpaisarn}

\author{
Chitchanok Chuengsatiansup\inst{1}
\thanks{This work was supported by the Netherlands Organisation for Scientific Research (NWO) under grant 639.073.005.}
\and
Vorapong Suppakitpaisarn\inst{2}
}

\institute{
Department of Mathematics and Computer Science \\
Technische Universiteit Eindhoven\\
P.O. Box 513, 5600 MB Eindhoven, the Netherlands\\
\email{c.chuengsatiansup@tue.nl}
\and
National Institute of Informatics\\
2-1-2 Hitotsubashi, Chiyoda-ku, Tokyo, 101-8430, Japan\\
\email{vorapong@nii.ac.jp}
}

\maketitle

\begin{abstract}
In this paper, we propose a fast algorithm for an elliptic curve multi-scalar multiplication based on the number representation called double-based number system (DBNS). As a method based on the representation is known to be one of the fastest algorithms up to date, DBNS is one of the tools used for reducing the computation time of a scalar multiplication.  A weak point of the DBNS-based scalar multiplication is its memory consumption. In the worst case, the method require us to store almost $n$ pre-computation points during its process.  We propose a method that can reduce the memory consumption in this paper. Using the same number of pre-computation points, the computation time of our multi-scalar multiplication is smaller than those of a sliding window method and a double-base chain representation. Beside a greedy heuristic algorithm, we also give an algorithm that can optimize the computation time in this setting. Our analysis shows that, when using the same number of pre-computation points, our optimal computation time is much smaller than the previous method. 

\smallskip
\textbf{Keywords:} 
elliptic curve, double-base number system, multi-scalar multiplication, interleaving method, optimal algorithm
\end{abstract}

\section{Introduction}

Multi-scalar multiplication is a bottleneck operation of several public- and symmetric-key cryptographic protocols,
including elliptic curve digital signature algorithm (ECDSA) \cite{ECDSA}.
Given positive integers $m, n$ and points on elliptic curves $P_1, P_2$,
we want to compute a point $R$ where
$R = m P_1 + n P_2.$

There are several proposed methods to compute such operation.
Most of those techniques precompute some elliptic points and store them in a memory.
In general, techniques with a larger number of precomputation points are usually faster than techniques with a smaller number of precomputation points.
However, those fast techniques cannot be used in the environment where only limited amount of memory is allowed.

By the number of precomputation points, we can categorize techniques proposed in the literature into two types.
The first type is techniques where the number of precomputation points does not depend on the size of scalars $m$ and $n$ such as Shamir’s trick \cite{Shamir},
interleaving method \cite{interleaving}, enlarged digit set \cite{enlarged2,enlarged4,enlarged1,enlarged3}, and double-base chain (DBC) \cite{dbc2,dbc3,DKS09}.
The second type is techniques where the number of precomputation points grows larger when scalars $m$ and $n$ are getting larger such as addition chain \cite{additionChain1,additionChain2}
and double-base number system (DBNS)  \cite{MH09}.

In this paper, we focus on multi-scalar multiplication using a technique which belongs to the second type.
To be more precisely, we extended the scalar multiplication, i.e., an operation to compute $R = nP$ where $n$ is a positive integer and $P$ is a point on an elliptic curve,
based on DBNS proposed by M\'eloni and Hasan in \cite{MH09}.
In their technique, an integer $r$ is written in a DBNS expansion $S_r \subseteq \mathbf{Z}_{\geq 0} \times \mathbf{Z}_{\geq 0}$ where
$$\sum_{(i,j) \in S_r} 2^i 3^j = r.$$ 
Since both $S^{(1)}_{41} = \{(2,2), (2,0), (0,0)\}$ and $S^{(2)}_{41} = \{(5,0), (0,2)\}$ are DBNS expansions of $41$, we know that a DBNS expansion of some $r$ is not unique.

Suppose $S_r$ is a DBNS expansion of $r$. Let $i_{\rm max}$ and $j_{\rm max}$ denote $\max\{ i' : (i',j') \in S_r \}$ and $\max\{ j' : (i',j') \in S_r \}$ respectively.
The method \cite{MH09} calculates and stores values $2^iP$ for all $i \in \{i : (i, j) \in S_r \}$ using point doubling.
Then, for each $j \in \{j : (i, j) \in S_r \}$, it calculates $d(j)P$ where $d(j) := \sum\limits_{i \in \{i’ : (i',j’) \in S_r\}} 2^i $ by point addition.
Therefore, $nP$ can be obtained by
$$nP = 3\cdot \left( \dots \left( 3 \cdot \left( 3 \cdot d(i_{\rm max})P + d(i_{\mathrm{max} - 1}P) \right) + d(i_{\mathrm{max} - 2})P \right) \dots \right) + d(0)P.$$
Using $S_{41}^{(1)}$ as an example, considering $i$ then points $2^0P$ and $2^2P$ are precomputed.
Considering $j$ then $d(0) = 2^2P + 2^0P = 5P$ for $(2,0),(0,0)$ and $d(2) = 2^2P = 4P$ for $(2,2)$ are also precomputed.
Therefore, $S_{41}^{(1)}$ can be obtained by
$$S_{41}^{(1)} = 3 \cdot \left( 3 \cdot d(2)P  \right) + d(0)P = 3 \cdot \left( 3 \cdot \left( 4P \right)  \right) + 5P = 41P.$$
%The computation requires $j_{\rm max} - 1$ triples, $i_{\rm max} - 1$ doubles, and $|S_n| - 1$ additions. --> No!!!
%precomp: i_{max}-1 doublings and in worst case i_{max} additions
%main comp: j_{max}-1 triplings and $|S_n} - 1$ additions
Since we have to store the points $2^iP$ for all $i \in \{i : (i,j) \in S_r\}$, the number of the precomputation points can be as large as $|S_r|$ in the worst case.

In \cite{dbns2}, a DBNS expansion of a given integer $r$ is obtained using a greedy algorithm.
Let $\mathcal{P}_r := \left\{ 2^i3^j : i,j \in \mathbf{Z} \text{ and } 2^i3^j \leq r \right\}$.
If $S^{(a)}_r$ is a DBNS expansion obtained by this algorithm,
it can be described as follows:
\[
S^{(a)}_r =
\begin{cases}
S^{(a)}_{r - 2^{i^*}3^{j^*}} \cup \{(i^*, j^*)\} & \text{if } 2^{i^*}3^{j*} := \max \mathcal{P}_r, \\
\emptyset       & \text{if } r = 0.
\end{cases}
\]
It is proved in the paper that, for any $r \in \mathbf{Z}_+$, we have $\left|S_r^{(a)}\right| \in O(\frac{\log r}{\log \log r})$.
Later, by the result in \cite{dbns3}, we know that the upper bound is already tight, i.e. $\left|S_r^{(a)}\right| \in \Theta(\frac{\log r}{\log \log r})$.

By the discussion in previous paragraphs, we need to store $\Theta(\frac{\log r}{\log \log r})$ precomputation points.
That is why we categorize the algorithm in \cite{MH09} to the second type of algorithms for the scalar multiplication.
In a computational environment with limited memory, we might not be able to store those $\Theta(\frac{\log r}{\log \log r})$ precomputation points.
Although a scalar multiplication method using DBNS is one of the fastest methods, we cannot use this method in such situation.

\subsection{Our Contributions}

In this paper, we propose an algorithm for multi-scalar multiplication that based on DBNS representation.
Compared to the DBNS-base scalar multiplication in \cite{MH09},
we significantly reduce the number of precomputation points that need to be stored.
Instead of storing all coefficients of $2^i$ and $d(j)$, we choose to store all coefficients of $3^j$.
We show that the computational time of the multi-scalar multiplication can be improved even when $j_{\rm max}$ is as small as $5$ or $6$.
Since the number of precomputation points is $2 \cdot j_{\rm max}$,
setting $j_{\rm max}$ to $5$ or $6$ makes our number of precomputation points approximately the same as those required in
double-base chain \cite{experiment,DKS09}, interleaving method \cite{interleaving}, and Shamir's trick \cite{enlarged4}.

The experimental results in Section \ref{sec:interleaving} show that our algorithm is faster than any proposed method in the literature.
For bit size $192$, $256$, $320$, $384$, $448$, and $512$,
our method is faster than the interleaving method by $1.5 - 2.0\%$
and faster than the tree-based double-base chain by $5.1 - 5.3\%$.

Suppose that the value of $j_{\rm max}$ is fixed.
For situations where we have enough time and memory to find a suitable $S_r$,
we propose a dynamic programming algorithm that finds a DBNS expansion $S^*_r$ of $r$, of which the size is smaller than any other DBNS expansions of $r$.
As discussed before, the number of point additions required for DBNS-based methods is equal to the size of the DBNS expansion.
Our algorithm reduces the computational time by reducing the size of the expansion.

% Don't quite understand the message that this paragraph tries to convey.
%It is possible that $S^*_n$ is not a DBNS expansion that can optimize the computational time,
%since there might be a DBNS expansion $S'_r$ with a larger number of point additions but much smaller number of point doublings (much smaller $i_{\rm max}$).
%However, it can be seen that the different between values $i_{\rm max}$ of two different DBNS expansions of $n$
%will not be more than $8$ or $9$ when $j_{\rm max}$ is equal to $5$ or $6$.
%By the fact that point doubling is not a time-consuming operation,
%we can say that the set $|S^*_n|$ can almost optimize the computation time of the DBNS-based multi-scalar multiplication.

In most cryptographic protocols, scalars $m$ and $n$ are usually private- or public-keys.
Generally, we perform several multi-scalar multiplications on embedded system using the same $m$ and $n$ after those numbers are generated on a personal computer.
Therefore, it is reasonable to spend time and memory to find the best DBNS expansion of $m$ and $n$ during the generation.

Motivated by the method in \cite{analysisMethod},
we propose an analysis that analyzes the average number of point additions in the optimal DBNS-based multi-scalar multiplication.
Given the same number of precomputation points, the analysis shows that our method requires much less point additions compared to
the interleaving method with fractional windows \cite{fractional} and the enlarged digit set \cite{analysisMethod}.

One might think that our algorithm is merely a double-and-add with a digit set $\{0,$ $\pm 3^0,$ $\pm 3^1,$ $\dots,$ $\pm 3^{j_{\rm max}}$ $\}$.
However, these two algorithms are not exactly the same.
Our algorithm allows more than one point additions to be performed between two doublings,
while the double-and-add allows only one point addition to be performed.

\newpage
\section{Interleaving Method}
To compute multi-scalar multiplication, e.g. $[m]P_1+[n]P_2$,
the simplest approach is to individually perform single-scalar multiplication
for $[m]P_1$ and $[n]P_2$ then add those results together.
If costs for computing $[m]P_1$ and $[n]P_2$ using double-and-add algorithm are
$d_1$ doublings plus $a_1$ additions and $d_2$ doubling plus $a_2$ additions respectively,
then the total cost for computing $[m]P_1 + [n]P_2$ is $d_1 + d_2$ doublings plus $a_1 + a_2 + 1$ additions.

Since the sum of $[m]P_1$ and $[n]P_2$ is needed, not the individual $[m]P_1$ and $[n]P_2$,
it is possible to perform doubling together for both $m$ and $n$
then followed by separate additions for $m$ and $n$.
This way, the total cost of interleaving method is $max(d_1,d_2)$ doublings plus $a_1 + a_2 + 1$ additions.

\example{Interleaving method for double-and-add} \\
Let $m = m_\ell \dots m_1 m_0 = {100101}_2$, $n = n_\ell \dots n_1 n_0 = {011001}_2$ and $R = [m]P_1 + [n]P_2$.
Let $m_i$ and $n_i$ be the $i$-bit of $m$ and $n$ respectively.
Starting by initializing $R$ to $[m_5]P_1 + [n_5]P_2 = [1]P_1 + [0]P_2] = P_1$, then iterate $i$ from $4$ down to $0$.
Each step perform $[2]R + [m_i]P_1 + [n_i]P_2$: \\
$i=4; \quad R = 2(P_1) + P_2 = 2P_1 + P_2$ \\
$i=3; \quad R = 2(2P_1 + P_2) + P_2 = 4P_1 + 3P_2$ \\
$i=2; \quad R = 2(4P_1 + 3P_2) + P_1 = 9P_1 + 6P_2$ \\
$i=1; \quad R = 2(9P_1 + 6P_2) = 18P_1 + 12P_2$ \\
$i=0; \quad R = 2(18P_1 + 12P_2) + P_1 + P_2 = 37P_1 + 25P_2$ \\

In this Section, we propose two algorithms based on interleaving method,
namely, interleaving signed sliding window and interleaving DBNS.
The former is to combine signed sliding window with interleaving method.
This algorithm is suitable for number represented in binary format.
The latter is to combine constrained DBNS with interleaving method.
This algorithm is suitable for number represented in binary-ternary format.



\subheading{Interleaving signed sliding window} \\
\label{sec:signedslide}
The speed leading algorithm for computing single-scalar multiplication is signed sliding window.
The idea of signed sliding window is to precompute a set $S = \{[1]P, [3]P, [5]P, \dots, [2^{\omega}-1]P\}$ where $\omega$ is the window size.
When scanning bits, instead of proceeding one bit and one doubling at a time,
now proceeding until the next non-zero bit is found, i.e., perform consecutive doublings,
then perform single addition $[c]P$ from set $S$ where $c$ is the longest consecutive bits within length $\omega$
starting from that non-zero bit and also ending with non-zero bit.

In this Subsection, we explain how signed sliding window can be extended to compute multi-scalar multiplication combining with interleaving method.
The algorithm is very straightforward.  Let use the same example of computing $[m]P_1 + [n]P_2$.
Similar to the interleaving method, doubling is performed on both $m$ and $n$ while addition is performed individually for $m$ and $n$.
However, we allow addition from precompute set $S_1 = \{[1]P_1, [3]P_1, \dots, [2^{\omega}-1]P_1\}$
and set $S_2 = \{[1]P_2, [3]P_2, \dots, [2^{\omega}-1]P_2\}$.
The searching process to find consecutive bits within length $\omega$ starting and ending with non-zero is perform in a similar manner as the sliding window.

Let ${\it{FindIndex}}(b,p)$ be a function performing a sliding window of width $\omega$ scanning a bit string $b$ from position $p$ to the right.
This function returns positions $i$ and $j$ closest to $p$ where $b_i$ and $b_j$ are 1.
The difference between $i$ and $j$, namely, $i-j$ must not exceed the window width $\omega$.
The value $i$ and $j$ may be the same if there is only one bit set within the width $\omega$.

To compute $[m]P_1 + [n]P_2$, $R$ is initialized to $0$ and function $FindIndex$ is called twice for $m$ and $n$.
Each bit in both $m$ and $n$ are scanned from left to right start at position $k = min(j_1,j_2)$ together with double $R$.
If $k$ is equal to $j_1$, then ${(m_{i_1},\dots,m_{{j_1}+1},m_{j_1})}_2$ is also addition to $R$ and $FindIndex(m,j_1-1)$ is called.
Similar to the case of $k$ equals $j_2$.
The algorithm is shown in Algorithm~\ref{interleaveSlidingAlgo}.

%%%%%%%%%%%%%%%%%%%%%%%%%%%%%%%%%%%%%%%%%%%%%%%%%%%%%%%%%%%%%%%%%%%%%%%%%%%%%%%%%%%%%%%%%%%%%%%%%%%%
% Algorithm
%%%%%%%%%%%%%%%%%%%%%%%%%%%%%%%%%%%%%%%%%%%%%%%%%%%%%%%%%%%%%%%%%%%%%%%%%%%%%%%%%%%%%%%%%%%%%%%%%%%%
\begin{algorithm}
\caption{Interleaving signed sliding window}
\begin{algorithmic}
	\Require Points $P_1,P_2$, sets $S_1=\{3P_1,5P_1,\dots,[2^{\omega}-1]P_1\}$,$S_2=\{3P_2,5P_2,\dots,[2^{\omega}-1]P_2\}$, \\
		and scalars $m={(m_{\ell_1},\dots,m_1,m_0)}_2$, $n={(n_{\ell_2},\dots,n_1,n_0)}_2$
	\Ensure Multi-scalar multiplication computing $[m]P_1 + [n]P_2$
	\Statex
	\State Initialize $R \gets 0$
	\State $(i_1,j_1)$ $\gets$ $FindIndex(m,\ell_1)$		\Comment{$FindIndex(b,p)$ returns ($i,j$) where $p {\ge} i {\ge} j, i{-}j {\le} \omega$ and $b_i{,}b_j {=} 1$}
	\State $(i_2,j_2)$ $\gets$ $FindIndex(n,\ell_2)$
	\If{$j_1 < j_2$} \State $k \gets j_1$
	\Else \State $k \gets j_2$
	\EndIf
	\While {$k$ $<$ len}			\Comment{len = maximum bit-length of $n_1$ and $n_2$}
		\State $R$ $\gets$ $[2]R$
		\If {$k = j_1$}
			\State Set $u$ $\gets$ ${(m_{i_1},\dots,m_{j_1+1},m_{j_1})}_2$
			\State $R$ $\gets$ $R + [u]P_1$		\Comment{$[u]P_1$ obtained from precomputation set $S_1$}
			\State $(i_1,j_1)$ $\gets$ $FindIndex(m,j_1{-}1)$
		\EndIf
		\If {$k = j_2$}
			\State Set $u$ $\gets$ ${(n_{i_2},\dots,n_{j_2+1},n_{j_2})}_2$
			\State $R$ $\gets$ $R + [u]P_2$		\Comment{$[u]P_2$ obtained from precomputation set $S_2$}
			\State $(i_2,j_2)$ $\gets$ $FindIndex(n,j_2{-}1)$
		\EndIf
		\State $k$ $\gets$ $k + 1$
	\EndWhile
	\\ \Return $[m]P_1 + [n]P_2$
\end{algorithmic}
\label{interleaveSlidingAlgo}
\end{algorithm}
%%%%%%%%%%%%%%%%%%%%%%%%%%%%%%%%%%%%%%%%%%%%%%%%%%%%%%%%%%%%%%%%%%%%%%%%%%%%%%%%%%%%%%%%%%%%%%%%%%%%

\example{Interleaving signed sliding window} \\
Let $m = m_\ell \dots m_1 m_0 = {100101}_2$, $n = n_\ell \dots n_1 n_0 = {011001}_2$, $R = [m]P_1 + [n]P_2$, and $\omega = 3$.
Let $m_{i_1,j_1}$ and $n_{i_2,j_2}$ be $m_{i_1} m_{i_1-1} \dots m_{j}$ and $n_{i_2} n_{i_2-1} \dots n_{j_2}$ respectively.
Scanning from $m_\ell$ and find $m_{i_1,j_1}$ such that $m_{i_1} = m_{j_1} = 1$ and $i_1 - j_1 \le \omega$.
All so do the same with $n$.
In this example, $m_{i_1,j_1} = m_{5,5}$ and $n_{i_2,j_2} = n_{4,3}$.
Because $j_1 > j_2$, $R$ is initialized to $[m_{5,5}]P_1 = P_1$
Then iterate $k$ from $4$ down to $0$.
Each step perform $[2]R$, if $j_1 = i$ add $[m_{i_1,j_1}]P_1$, if $j_2 = i$ add $[n_{i,j_2}]P_2$: \\
$k=4; \quad R = 2(P_1) = 2P_1$ \\
$k=3; \quad R = 2(2P_1)= 4P_1$ \\
$k=2; \quad R = 2(4P_1) + 3P_2 = 9P_1 + 6P_2$ \\
$k=1; \quad R = 2(9P_1 + 6P_2) = 18P_1 + 12P_2$ \\
$k=0; \quad R = 2(18P_1 + 12P_2) + P_1 + P_2 = 37P_1 + 25P_2$ \\


To illustrate the performance of this interleaving signed sliding window algorithm,
we conducted experiments using randomly chosen $10000$ pairs of integers $m$ and $n$ per each bit range 192, 256, 320, 384, 448, and 512.
We tried different window widths range from $0$ to $10$.
The performance is measured by the number of multiplication ($N_M$) required to compute $[m]P_1 + [n]P_2$.
We use mixed coordinate systems, namely, projective twisted Edwards and extend twisted Edwards with $a=-1$
in order to gain speed from both doubling and addition.
With this curve choice, a doubling takes $3\mul + 4\sqr$, a regular addition takes $8\mul$, and a mixed addition takes $7\mul$
where $\mul$ and $\sqr$ denote field multiplication and field squaring respectively.
We also use a common assumption that $\sqr \approx 0.8\mul$.
Note that we perform conversion in order to make precomputations in affine so that the mixed addition formulas can be used.
The cost displays in Table~\ref{signedslideTable} does include the cost of precomputation but exclude the cost of conversion.
The notation {$\mathcal{\#P}$} denotes the number of precomputation points.


%%%%%%%%%%%%%%%%%%%%%%%%%%%%%%%%%%%%%%%%%%%%%%%%%%%%%%%%%%%%%%%%%%%%%%%%%%%%%%%%%%%%%%%%%%%%%%%%%%%%
% Table
%%%%%%%%%%%%%%%%%%%%%%%%%%%%%%%%%%%%%%%%%%%%%%%%%%%%%%%%%%%%%%%%%%%%%%%%%%%%%%%%%%%%%%%%%%%%%%%%%%%%
\begin{table}[h]
\centering
\begin{tabular}{|C{0.1\textwidth}| *6{C{0.06\textwidth} C{0.06\textwidth}|} }
%\begin{tabularx}{\textwidth}{|X| *6{X X|}}
\toprule
%\hline
\multirow{2}{*}{width $\omega$}
	&\multicolumn{2}{c|}{192-bit}
		&\multicolumn{2}{c|}{256-bit}
			&\multicolumn{2}{c|}{320-bit}
				&\multicolumn{2}{c|}{384-bit}
					&\multicolumn{2}{c|}{448-bit}
						&\multicolumn{2}{c|}{512-bit} \\
	&\tiny{$N_M$}	&\tiny{$\mathcal{\#P}$}
		&\tiny{$N_M$}	&\tiny{$\mathcal{\#P}$}
			&\tiny{$N_M$}	&\tiny{$\mathcal{\#P}$}
				&\tiny{$N_M$}	&\tiny{$\mathcal{\#P}$}
					&\tiny{$N_M$}	&\tiny{$\mathcal{\#P}$}
						&\tiny{$N_M$}	&\tiny{$\mathcal{\#P}$} \\
\midrule
1 &2103 &2 &2798 &2 &3494 &2 &4189 &2 &4885 &2 &5580 &2 \\
2 &2103 &2 &2798 &2 &3494 &2 &4189 &2 &4885 &2 &5580 &2 \\
3 &1884 &6 &2494 &6 &3104 &6 &3714 &6 &4324 &6 &4935 &6 \\
4 &1838 &10 &2424 &10 &3009 &10 &3595 &10 &4181 &10 &4767 &10 \\
5 &1836 &20 &2391 &21 &2946 &22 &3500 &22 &4056 &22 &4611 &22 \\
6 &1932 &29 &2478 &33 &3016 &36 &3550 &38 &4084 &39 &4618 &40 \\
7 &2049 &36 &2629 &43 &3195 &50 &3752 &55 &4300 &60 &4839 &63 \\
8 &2183 &37 &2786 &47 &3374 &56 &3952 &65 &4520 &72 &5082 &80 \\
9 &2361 &36 &3008 &47 &3631 &57 &4239 &67 &4837 &77 &5425 &86 \\
10 &2531 &34 &3238 &45 &3919 &55 &4571 &65 &5215 &75 &5841 &85 \\

%\hrule
\bottomrule
\multicolumn{13}{c}{}
%\end{tabularx}
\end{tabular}
\caption{Number of multiplications and precomputation points for different window widths to compute double-scalar multiplication using interleaving signed sliding window}
\label{signedslideTable}
\end{table}
%%%%%%%%%%%%%%%%%%%%%%%%%%%%%%%%%%%%%%%%%%%%%%%%%%%%%%%%%%%%%%%%%%%%%%%%%%%%%%%%%%%%%%%%%%%%%%%%%%%%




\subheading{Interleaving DBNS} \\
In cases that tripling operation is also allowed, scalars can be represented using $\{2-3\}$ Double-Base Number System (DBNS).
$$n = \Sigma^{\ell}_{i=1} c_i 2^{a_i}3^{b_i}$$
The main advantage of DBNS is that there are fewer terms in the number expansion compared to binary representation.
However, the main drawback of DBNS is that scalar multiplication cannot be performed using Hornor-like method.
Therefore, the cost of doubling is at least $max(a_i)$.  Similarly, the cost of tripling is at least $max(b_i)$.
In \cite{DIM05}, a restrictive DBNS called {\it{double-base chain}} (DBC) was introduced.
Exponents $a_i$ and $b_i$ in the expansion can no long be freely chosen and have to obey the restrictions
$a_\ell \ge a_{\ell-1)} \ge \cdots \ge a_1$ and $b_\ell \ge b_{\ell-1} \ge \cdots \ge b_1$.
These extra conditions on the exponents increase the number of terms in the expansion.
With these restrictions, Hornor-like method can be applied to DBC when computing scalar multiplication, e.g.,
$[841232]P = [3]([3]([3]([2^13^3]([2^63^2]P+P)-P)-P)+P)-P$.
Unlike DBNS, the cost of doubling and tripling for DBC is guaranteed to be at most $max(a_i)$ and $max(b_i)$.

In \cite{MH09}, M\'eloni and Hasan proposed an algorithm using DBNS by precomputing a set containing power of $2$ and
performing scalar multiplication using Hornor-like method on power of $3$.
The drawback of this algorithm is the amount of precomputation points.

Inspried by that idea, we took a similar approach of using DBNS and try to make it applicable for Hornor-like method.
Instead of forcing both exponents $a_i$ and $b_i$ to be in decreasing order,
we force only exponent $a_i$ to be in decreasing order but force exponent $b_i$ to not exceed a parameter $b^{max}$.
In other words, the restriction becomes
$a_\ell \ge a_{\ell-1} \ge \cdots \ge a_1$ (same as DBC) and $b_1,b_2,\dots,b_\ell \le m$.
This way, Hornor-like method can now applied to exponent $a_i$.

\example{Apply Hornor-like method to DBNS on single exponent to compute scalar multiplication} \\
$[983828]P = [2^43^{10}]P - [2^23^4]P + [2^13^9]P + [2^1]P$ \\
$\phantom{[983828]P} = [2]([2^33^{10}]P - [2^13^4]P + [3^9]P + P)$ \\
$\phantom{[983828]P} = [2]([2]([2^23^{10}]P - [3^4]P) + [3^9]P + P)$ \\
Note that $[3^4]P, [3^9]P, [3^{10}]P$ need to be precomputed. \\

In this Subsection, we explain how to combine the new restriction DBNS with the interleaving method to compute multi-scalar multiplication.
The essential part to compute $[m]P_1 + [n]P_2$ giving $b^{max}$ a bound on maximum power of 3 is find a good $\{2,3\}$ DBNS for $m$ and $n$.
At each step, we find two approximations, namely, one closest $m$ and the other one closest to $n$.
Then we choose the best one between those two the give the best approximation to both $m$ and $n$ and subtract from them.
Keep iterating these steps until $m$ and $n$ reach $0$.
This algorithm requires precomputation of $2 \times b^{max}$ points using $2 \times b^{max}$ triplings.
The main computation consists of only doublings and additions.
Algorithm~\ref{interleaveDBNSAlgo} shows the interleaving DBNS.

\example{Generating chain of interleaving DBNS} \\
Let $m = 554627$, $n = 748556$ and $b^{max} = 5$.
Let $(a_i,b_i)$ where $b_i \le 5$ make $2^{a_i}3^{b_i}$ the best approximation of $m$ and $n$ and step $i$. \\
$(a_0,b_0) = (18,1)$;		$\quad n\phantom{m} = \phantom{-}748556			- 2^{18}3^{1}		= -37876$ \\
$(a_1,b_1) = (19,0)$;		$\quad m\phantom{n} = \phantom{-}554627			- 2^{19}3^{0}		= \phantom{-}30339$ \\
$(a_2,b_2) = (12,2)$;		$\quad n\phantom{m} = -37876\phantom{0}			+ 2^{12}3^{2}		= -1012$ \\
$(a_3,b_3) = (\phantom{0}7,5)$; $\quad m\phantom{n} = \phantom{-}30339\phantom{0}	- 2^{\phantom{0}7}3^{5} = -765$ \\
$(a_4,b_4) = (10,0)$;		$\quad n\phantom{m}=  -1012\phantom{00}			+ 2^{10}3^{0}		= \phantom{-}12$ \\
$(a_5,b_5) = (\phantom{0}8,1)$; $\quad m\phantom{n}=  -765\phantom{000}			+ 2^{\phantom{0}8}3^{1} = \phantom{-}3$ \\
$(a_6,b_6) = (\phantom{0}2,1)$; $\quad n\phantom{m}= \phantom{-}12\phantom{0000}	- 2^{\phantom{0}2}3^{1} = \phantom{-}0$ \\
$(a_7,b_7) = (\phantom{0}0,1)$; $\quad m\phantom{n}= \phantom{-}3\phantom{00000}	- 2^{\phantom{0}0}3^{1} = \phantom{-}0$ \\
Therefore, an interleaving DBNS chain for $[554627]P_1 + [748556]P_2$ is
$[2^{18}3^{1}]P_2 + [2^{19}3^{0}]P_1 - [2^{12}3^{2}]P_2 + [2^{7}3^{5}]P_1 - [2^{10}]P_2 - [2^{8}3^{1}]P_1 + [2^{2}3^{1}]P_2 + [2^{0}3^{1}]P_1$.
Apply Hornor-like method to the power of $2$, then $[554627]P_1 + [748556]P_2$ can be computed as
$[2^2]([2^5]([2]([2^2]([2^2]([2^6]([2]P_1 + [3^1]P_2) - [3^2]P_2) - [3^0]P_2) - [3^1]P_1) + [3^5]P_1) + [3^1]P_2) + [3^1]P_1$.
Note: points need to be precomputed are: $[3]P_1, [3^5]P_1, [3]P_2, [3^2]P_2$. \\


%%%%%%%%%%%%%%%%%%%%%%%%%%%%%%%%%%%%%%%%%%%%%%%%%%%%%%%%%%%%%%%%%%%%%%%%%%%%%%%%%%%%%%%%%%%%%%%%%%%%
% Algorithm
%%%%%%%%%%%%%%%%%%%%%%%%%%%%%%%%%%%%%%%%%%%%%%%%%%%%%%%%%%%%%%%%%%%%%%%%%%%%%%%%%%%%%%%%%%%%%%%%%%%%
\begin{algorithm}
\caption{Interleaving DBNS}
\begin{algorithmic}
%	\Require Points $P_1,P_2$, sets $S_1=\{3P_1,9P_1,\dots,[3^{b_{max}}]P_1\}$,$S_2=\{3P_2,9P_2,\dots,[3^{b_{max}}]P_2\}$, \\
%		and scalars $m={(m_{\ell_1},\dots,m_1,m_0)}_2$, $n={(n_{\ell_2},\dots,n_1,n_0)}_2$
%	\Ensure Multi-scalar multiplication computing $[m]P_1 + [n]P_2$
	\Require Scalars $m, n$ and a bound $b^{max}$ for the maximum power of $3$
	\Ensure	A $\{2,3\}$-DBNS representing $m+n$ such that the maximum power of $3$ does not exceed $b^{max}$
	\Statex
	\State Initialize $R \gets 0$
	\While {$m \ne 0$ or $n \ne 0$}
		\State $(a,b) \gets FindTerm(m,n)$
		\State $m \gets m - 2^a3^b$
		\State $n \gets n - 2^a3^b$
	\EndWhile
	\\ \Return $[m]P_1 + [n]P_2$
\end{algorithmic}

\begin{algorithmic}
	\Statex
	\Statex
%	\Require Scalar $r_1,r_2$ and a bound $b^{max}$ for the maximum power of $3$
%	\Ensure A pair $(a,b)$ making $2^a3^b$ the best approximation of $m$ and $n$
	\Statex
	\State \bf{FindTerm($r_1,r_2$):}
	%\State Initialize $r \gets 0$
	\For {$0 \le a_i \le a^{max}$ and $0 \le b_i \le b^{max}$}
		\State $d_1 \gets r_1 - 2^{a_i}3^{b_i}$
		\If {$d_1 < min_1$}
			\State $min_1 \gets d_1$
			\State $pair_1 \gets (a_i,b_i)$
		\EndIf
		\State $d_2 \gets r_2 - 2^{a_i}3^{b_i}$
		\If {$d_2 < min_2$}
			\State $min_2 \gets d_2$
			\State $pair_2 \gets (a_i,b_i)$
		\EndIf
	\EndFor
	\If {$d_1 < d_2$}
		$(a,b) \gets pair_1$
	\Else
		$(a,b) \gets pair_2$
	\EndIf
	\\ \Return $(a,b)$
\end{algorithmic}

\label{interleaveDBNSAlgo}
\end{algorithm}
%%%%%%%%%%%%%%%%%%%%%%%%%%%%%%%%%%%%%%%%%%%%%%%%%%%%%%%%%%%%%%%%%%%%%%%%%%%%%%%%%%%%%%%%%%%%%%%%%%%%


Table~\ref{dbnsTable} shows the performance of the interleaving DBNS algorithm.
In this experiment, we also tested different $b^{max}$ values from $0$ to $15$.
The setting was similar to the experiment on interleaving signed sliding window ~\ref{sec:signedslide},
namely, we used $10000$ randomly chosen pairs of integers $m$ and $n$ for bit range 192, 256, 320, 384, 448 an 512.
We use the same mixed coordinate systems projective and extended twisted Edwards with $a=-1$.
Since we use tripling for precomputation points which will be added, we need to output tripling in extend coordinates.
Therefore tripling takes $11\mul + 3\sqr$.  Again, we use a common assumption that $\sqr \approx 0.8\mul$.
Note again that we perform conversion in order to make precomputations in affine so that the mixed addition formulas can be used.
The cost displays in Table~\ref{dbnsTable} does include the cost of precomputation but exclude the cost of conversion.
The notation {$\mathcal{\#P}$} denotes the number of precomputation points.

At 192- and 256-bit, $b^{max} = 6$ or having $12$ precomputation points gives the best performance.
The $b^{max}$ value increase as the bit size increase, meaning that more space to keep precomputation points is required.

%%%%%%%%%%%%%%%%%%%%%%%%%%%%%%%%%%%%%%%%%%%%%%%%%%%%%%%%%%%%%%%%%%%%%%%%%%%%%%%%%%%%%%%%%%%%%%%%%%%%
% Table
%%%%%%%%%%%%%%%%%%%%%%%%%%%%%%%%%%%%%%%%%%%%%%%%%%%%%%%%%%%%%%%%%%%%%%%%%%%%%%%%%%%%%%%%%%%%%%%%%%%%
\begin{table}[h]
\centering
\begin{tabular}{|C{0.1\textwidth}| *6{C{0.06\textwidth} C{0.06\textwidth}|} }
%\begin{tabularx}{\textwidth}{|X| *6{X X|}}
\toprule
%\hline
\multirow{2}{*}{$b^{max}$}
	&\multicolumn{2}{c|}{192-bit}
		&\multicolumn{2}{c|}{256-bit}
			&\multicolumn{2}{c|}{320-bit}
				&\multicolumn{2}{c|}{384-bit}
					&\multicolumn{2}{c|}{448-bit}
						&\multicolumn{2}{c|}{512-bit} \\
	&\tiny{$N_M$}	&\tiny{$\mathcal{\#P}$}
		&\tiny{$N_M$}	&\tiny{$\mathcal{\#P}$}
			&\tiny{$N_M$}	&\tiny{$\mathcal{\#P}$}
				&\tiny{$N_M$}	&\tiny{$\mathcal{\#P}$}
					&\tiny{$N_M$}	&\tiny{$\mathcal{\#P}$}
						&\tiny{$N_M$}	&\tiny{$\mathcal{\#P}$} \\
\midrule
0 &2075 &0 &2761 &0 &3464 &0 &4167 &0 &4854 &0 &5534 &0 \\
1 &1917 &2 &2566 &2 &3192 &2 &3845 &2 &4470 &2 &5098 &2 \\
2 &1894 &4 &2507 &4 &3135 &4 &3764 &4 &4379 &4 &4974 &4 \\
3 &1872 &6 &2464 &6 &3076 &6 &3687 &6 &4281 &6 &4868 &6 \\
4 &1836 &8 &2417 &8 &3008 &8 &3586 &8 &4196 &8 &4766 &8 \\
5 &1836 &10 &2410 &10 &3000 &10 &3587 &10 &4170 &10 &4755 &10 \\
6 &1828 &12 &2396 &12 &2983 &12 &3553 &12 &4124 &12 &4708 &12 \\
7 &1837 &14 &2410 &14 &2983 &14 &3544 &14 &4108 &14 &4686 &14 \\
8 &1843 &16 &2407 &16 &2980 &16 &3540 &16 &4105 &16 &4671 &16 \\
9 &1854 &18 &2412 &18 &2976 &18 &3540 &18 &4084 &18 &4656 &18 \\
10 &1859 &20 &2421 &20 &2974 &20 &3530 &20 &4080 &20 &4633 &20 \\
11 &1865 &22 &2422 &22 &2971 &22 &3523 &22 &4069 &22 &4623 &22 \\
12 &1878 &24 &2435 &24 &2973 &24 &3537 &24 &4082 &24 &4629 &24 \\
13 &1883 &26 &2438 &26 &2989 &26 &3543 &26 &4092 &26 &4635 &26 \\
14 &1901 &28 &2444 &28 &3000 &28 &3549 &28 &4092 &28 &4647 &28 \\
15 &1916 &30 &2467 &30 &3018 &30 &3553 &30 &4097 &30 &4653 &30 \\

%\hrule
\bottomrule
\multicolumn{13}{c}{}
%\end{tabularx}
\end{tabular}
\caption{Number of multiplications and precomputation points for different $b^{max}$ to compute double-scalar multiplication using interleaving DBNS}
\label{dbnsTable}
\end{table}
%%%%%%%%%%%%%%%%%%%%%%%%%%%%%%%%%%%%%%%%%%%%%%%%%%%%%%%%%%%%%%%%%%%%%%%%%%%%%%%%%%%%%%%%%%%%%%%%%%%%

\newpage

\section{Optimal Algorithm}

add text here

\subheading{This is subheading}
add more text

\section{Conclusion}

add conclusion here



\clearpage
\bibliographystyle{plain}
\bibliography{collection}

\end{document}
