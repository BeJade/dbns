\documentclass[a4paper,11pt,runningheads]{llncs}

%\usepackage[margin=2.7cm]{geometry}

\pdfpagewidth=210 true mm
\pdfpageheight=297 true mm

\usepackage[table]{xcolor}
\definecolor{linkcolor}{rgb}{0.65,0,0}
\definecolor{citecolor}{rgb}{0,0.65,0}
\definecolor{urlcolor}{rgb}{0,0,0.65}
\usepackage[breaklinks=true,colorlinks=true, backref=page, linkcolor=linkcolor, urlcolor=urlcolor, citecolor=citecolor]{hyperref}
\usepackage{hyperref}

\usepackage{amsfonts,amsmath,amscd,amssymb,array}
\usepackage{algorithm, algpseudocode}
\usepackage{tabularx,booktabs}
\usepackage{array}
\usepackage{multirow}
\usepackage{float}
\floatname{listing}{Listing}

\def\subheading#1{\medskip\refstepcounter{subsection}\noindent{\boldmath\textbf{\thesubsection. #1.} }\ignorespaces}	% with number ex. 1.1 Subheading
%\def\subheading#1{\medskip\noindent{\boldmath\textbf{#1}}~\ignorespaces}	% without number ex. Subheading

\def\mul{\text{\bf M}}
\def\sqr{\text{\bf S}}
\def\con{\text{\bf m}}

\newcolumntype{L}[1]{>{\raggedright\let\newline\\\arraybackslash\hspace{0pt}}m{#1}}
\newcolumntype{C}[1]{>{\centering\let\newline\\\arraybackslash\hspace{0pt}}m{#1}}

\renewcommand{\algorithmicrequire}{\textbf{Input:}}
\renewcommand{\algorithmicensure}{\textbf{Output:}}

%\def\example#1{\medskip\noindent{\bolmath\textbf{#1}}~\ignorespaces}

%\def\setof#1{\mathord{\left\lbrace{#1}\right\rbrace}}
%\def\floor#1{\mathord{\left\lfloor{#1}\right\rfloor}}
%\def\min#1{\operatorname{min}\mathopen{}\setof{#1}}
%\def\max#1{\operatorname{max}\mathopen{}\setof{#1}}
%\def\Z{\mathord{\text{\bf Z}}}
%\def\F{\mathord{\text{\bf F}}}
%\def\F{\mathbb{F}}

%\numberwithin{table}{section}
%\numberwithin{figure}{section}
%\numberwithin{theorem}{section}

%\def\dash{$\,$---$\,$}

%\makeatletter
%\renewcommand\labelitemi{$\m@th\bullet$}
%\let\c@figure\c@subsection
%\let\c@table\c@subsection
%\makeatother

%\spnewtheorem{mytheorem}[subsection]{Theorem}{\bfseries}{\itshape}




\begin{document}

\titlerunning{Multi-Scalar Multiplication Using DBNS }
\title{Fast Elliptic Curve Multi-Scalar Multiplication Using DBNS with Constant Number of Pre-Computation Points}
\authorrunning{C. Chuengsatiansup, V. Suppakitpaisarn}

\author{
Chitchanok Chuengsatiansup\inst{1}
\thanks{This work was supported by the Netherlands Organisation for Scientific Research (NWO) under grant 639.073.005.}
\and
Vorapong Suppakitpaisarn\inst{2}
}

\institute{
Department of Mathematics and Computer Science \\
Technische Universiteit Eindhoven\\
P.O. Box 513, 5600 MB Eindhoven, the Netherlands\\
\email{c.chuengsatiansup@tue.nl}
\and
National Institute of Informatics\\
2-1-2 Hitotsubashi, Chiyoda-ku, Tokyo, 101-8430, Japan\\
\email{vorapong@nii.ac.jp}
}

\maketitle

\begin{abstract}
In this paper, we propose a fast algorithm for an elliptic curve multi-scalar multiplication based on the number representation called double-based number system (DBNS). As a method based on the representation is known to be one of the fastest algorithms up to date, DBNS is one of the tools used for reducing the computation time of a scalar multiplication.  A weak point of the DBNS-based scalar multiplication is its memory consumption. In the worst case, the method require us to store almost $n$ pre-computation points during its process.  We propose a method that can reduce the memory consumption in this paper. Using the same number of pre-computation points, the computation time of our multi-scalar multiplication is smaller than those of a sliding window method and a double-base chain representation. Beside a greedy heuristic algorithm, we also give an algorithm that can optimize the computation time in this setting. Our analysis shows that, when using the same number of pre-computation points, our optimal computation time is much smaller than the previous method. 

\smallskip
\textbf{Keywords:} 
double-base number system, single-scalar multiplication, multi-scalar multiplication
\end{abstract}

\section{Introduction}

Multi-scalar multiplication is a bottleneck operation of several public and private cryptographic protocols, including elliptic curve digital signature algorithm (ECDSA) \cite{ECDSA}. Given positive integers $r_1, r_2$ and points on elliptic curves $P, Q$, we want to compute a point $S$ when 
$$S = r_1 P + r_2 Q.$$

There are several methods proposed to compute the operation. In most of those techniques, we pre-compute some elliptic points, and store them in our memory. In general, techniques with larger numbers of pre-computation points are usually faster than techniques with a smaller number. However, we cannot use those fast techniques in a computation environment with a limited amount of memory. 

By the number of pre-computation points, we can divide techniques proposed in literature into two types. The first type is techniques when the number of pre-computation points does not depend on the size of $r_1$ and $r_2$ such as Shamir’s trick \cite{Shamir}, interleaving method \cite{interleaving}, enlarged digit set \cite{enlarged2,enlarged4,enlarged1,enlarged3}, and double-base chain (DBC) \cite{dbc2,dbc3,dbc1}. The second type is techniques when the number of the points will be larger when $r_1$ and $r_2$ is larger such as addition chain \cite{additionChain1,additionChain2} and double-base number system (DBNS)  \cite{dbns}.

In this paper, we will focus on the multi-scalar multiplication based on DBNS. The technique for a scalar multiplication, an operation to compute $S = rP$ when $r$ is a positive integer and $P$ is an elliptic point, proposed by Meloni and Hasan in \cite{dbns}.  In the technique, the integer $r$ will be written in DBNS form $S_r \subseteq \mathbf{Z}_{\geq 0} \times \mathbf{Z}_{\geq 0}$ where
$$\sum_{(i,j) \in S_r} 2^I 3^j = r.$$ 
Since both $S^{(1)}_{41} = \{(2,2), (2,0), (0,0)\}$ and $S^{(2)}_{41} = \{(5,0), (0,2)\}$ are the DBNS form of 41, we know that the DBNS form of some $r$ is not unique.


Suppose that we have a DBNS form of $r$, $S_r$. Let $i_{\rm max}$ and $j_{\rm max}$ denote $\max\{ i' : (i',j') \in S_r \}$ and $\max\{ j' : (i',j') \in S_r \}$ respectively. The method calculates and stores the value of $3^iP$ for all $j \in \{j’ : (i’, j’) \in S_r \}$ using point triples. Then, for each $i \in \{i’ : (i’, j’) \in S_r \}$, it calculate $d(i)P$ where $d(i) := \sum\limits_{j \in \{j’ : (i,j’) \in S_r\}} 3^j $ by point additions. By those pre-computation, the point $rP$ can be obtained by
$$rP = 2\cdot \left( \dots \left( 2 \cdot \left( 2 \cdot d(i_{\rm max})P + d(i_{\mathrm{max} - 1}P) \right) + d(i_{\mathrm{max} - 2})P \right) \dots \right) + d(0)P.$$
For example, using $S_{41}^{(1)}$, we have $i_{\rm max} = j_{\rm max} = 2$, $d(2) = \left(3^2 + 3^0\right) = 10$, and $d(0) =  3^0 = 1$. The point $41P$ can be obtained by
$$41P = 2 \cdot \left( 2 \cdot d(2)P  \right) + d(0)P = 2 \cdot \left( 2 \cdot \left( 10P \right)  \right) + P.$$
The computation requires $j_{\rm max} - 1$ triples, $i_{\rm max} - 1$ doubles, and $|S_r| - 1$ additions. Since we have to store the points $d(i)P$ for all $i \in \{i : (i,j) \in S_r\}$, the number of the precomputation points can be as large as $|S_r|$ in the worst case.

In \cite{dbns2}, the DBNS form of a given integer $r$ is found using a greedy algorithm. Let $\mathcal{P}_r := \left\{ 2^i3^j : i,j \in \mathbf{Z} \text{ and } 2^i3^j \leq r \right\}$. When $S^{(a)}_r$ is the DBNS form found the algorithm, the set can be described as follows:
\[
S^{(a)}_r =
\begin{cases}
S^{(a)}_{r - 2^{i^*}3^{j^*}} \cup \{(i^*, j^*)\} & \text{if } 2^{i^*}3^{j*} := \max \mathcal{P}_r, \\
\emptyset       & \text{if } r = 0.
\end{cases}
\]
It is proved in the paper that, for any $r \in \mathbf{Z}_+$, we have $\left|S_r^{(a)}\right| \in O(\frac{\log r}{\log \log r})$. Later, by the result in \cite{dbns3}, we know that the upper bound is already tight, i.e. $\left|S_r^{(a)}\right| \in \Theta(\frac{\log r}{\log \log r})$.

By the discussion in the previous paragraphs, we need to store $\Theta(\frac{\log r}{\log \log r})$. That is why we categorize an algorithm in \cite{dbns} to the second type of the algorithms for the multi-scalar multiplication. In a computation environment with limited memory, we might not be able to store those $\Theta(\frac{\log r}{\log \log r})$ pre-computation points. Although a scalar multiplication method using DBNS is one of the fastest method, we cannot use the method in that situation.

\subsection{Our Contributions}

In this paper, we propose an algorithm that can reduce the number of pre-computation points of multi-scalar multiplication based on DBNS. Instead of storing all coefficients of $2^i$, $d(i)$, we choose to store all coefficients of $3^j$ in this paper. We show that the computation time of the multi-scalar multiplication can be significantly improved even when $j_{\rm max}$ is as small as $5$ or $6$. Since the number of pre-computation points is $2 \cdot j_{\rm max}$, setting $j_{\rm max}$ to $5$ or $6$ makes our number of pre-computation points equals those required in double-base chain \cite{experiment,dbc1}, interleaving method \cite{interleaving}, and Shamir's trick \cite{enlarged4}.

With the same number of pre-computation points, the experimental results in Section \ref{Section:Interleaving} shows that our algorithm is faster than any proposed method in literature when the size of $r$ is larger than $256$ bits. Our method is faster than the interleaving method by $???\%$, $???\%$, $???\%$, $???\%$, and $???\%$ when the number of bits is $256$, $320$, $384$, $448$, and $512$ respectively. Besides, the improvement over the tree-based double-base chain is are  $???\%$, $???\%$, $???\%$, $???\%$, and $???\%$ when the number of bits is $256$, $320$, $384$, $448$, and $512$.

Suppose that the value of $j_{\rm max}$ is fixed. For the situation when we have enough time and memory to find a suitable $S_r$, we propose a dynamic programming algorithm that can find a DBNS form of $r$, $S^*_r$, of which size is smaller than any other DBNS of $r$. As discussed before, the number of point additions required for the DBNS-based method is equal to the size of DBNS form. We can reduce the computation time by reducing the size of the form using our algorithm.

It is possible that $S^*_r$ is not a DBNS form that can optimize our computation time, since there might be a DBNS form $S'_r$ with a larger number of point additions but much smaller number of point doubles (much smaller $i_{\rm max}$). However, it can be seen that the different between the values of $i_{\rm max}$ of two different DBNS form of $r$ will not be more than $8$ or $9$, when $j_{\rm max}$ is equal to $5$ or $6$. By the fact that point doubles is not a time-consuming operation, we can say that the set $|S^*_r|$ can almost optimize the computation time of the DBNS-based multi-scalar multiplication.

We note that the time and memory consumption of our dynamic algorithm do not increase the cost of the multi-scalar multiplication. In most of cryptographic protocol, the scalars $r_1$ and $r_2$ are usually a private or public key. Generally, we perform several multi-scalar multiplications on embedded system using the same $r_1$ and $r_2$, after we generated those numbers on a personal computer. Therefore, it is reasonable to take time and memory to find the best DBNS form of $r_1$ and $r_2$ during their generation.

Motivated by the method in \cite{analysisMethod}, we propose an analysis method that can analyze the average number of point additions in the optimal DBNS-based multi-scalar multiplication. Given the same number of pre-compuation points, the analysis shows that our method requires much less point additions compared to the interleaving method with fractional windows \cite{fractional} or the enlarged digit set \cite{analysisMethod}.

One might think that our algorithm is merely a double-and-add technique with a digit set $\left\{0, \pm 3^0, \pm 3^1, \dots, \pm 3^{j_{\rm max}} \right\}$. However, our method is different from that technique. In the technique, only one point addition will be performed between two sets of point doubles, while more than one point additions are allowed to perform between those two sets.

\section{Interleaving Method}
To compute multi-scalar multiplication, e.g. $[m]P_1+[n]P_2$,
the simplest approach is to individually perform two single-scalar multiplications,
one for $[m]P_1$ and one for$[n]P_2$, then add those results together.
If the cost for computing single-scalar multiplication using double-and-add algorithm
for $[m]P_1$ is $d_1$ doublings plus $a_1$ additions and for $[n]P_2$ is $d_2$ doubling plus $a_2$ additions,
it is easy to see that the total cost for computing $[m]P_1 + [n]P_2$ is $d_1 + d_2$ doublings plus $a_1 + a_2 + 1$ additions.

Since what needed to be computed is the sum of $[m]P_1$ and $[n]P_2$, not the individual $[m]P_1$ and $[n]P_2$,
a better way to compute this multi-scalar multiplication is to mutually perform doubling on both $m$ and $n$
then followed by individual additions for $m$ and $n$.
This interleaving method reduces the total cost to compute $[m]P_1 + [n]P_2$ down to
$max(d_1,d_2)$ doublings plus $a_1 + a_2 + 1$ additions.
See \cite{DI08} for more details.

\example{Interleaving double-and-add} \\
Let $m = m_\ell \dots m_1 m_0 = {100101}_2$ and $n = n_\ell \dots n_1 n_0 = {011001}_2$.
The $i$-bit of $m$ and $n$ denote by $m_i$ and $n_i$ respectively.
Let the result of $[m]P_1 + [n]P_2$ be kept in $R$.
The computation starts by initializing $R$ to $[m_5]P_1 + [n_5]P_2 = [1]P_1 + [0]P_2 = P_1$.
Then iterate $i$ from $4$ down to $0$, and perform $[2]R + [m_i]P_1 + [n_i]P_2$ at each step if $[m_i]$ or $[n_i]$ is non-zero: \\
$i=4; \quad R = 2(P_1) + [1]P_2 = 2P_1 + P_2$ \\
$i=3; \quad R = 2(2P_1 + [1]P_2) + P_2 = 4P_1 + 3P_2$ \\
$i=2; \quad R = 2(4P_1 + 3P_2) + [1]P_1 = 9P_1 + 6P_2$ \\
$i=1; \quad R = 2(9P_1 + 6P_2) = 18P_1 + 12P_2$ \\
$i=0; \quad R = 2(18P_1 + 12P_2) + [1]P_1 + [1]P_2 = 37P_1 + 25P_2$ \\

In this Section, we introduce two algorithms based on the interleaving method,
namely, interleaving signed sliding window and interleaving Double-Base Number System (DBNS).
The former is to combine signed sliding window with interleaving method.
This algorithm is suitable for number represented in binary format.
The latter is to combine constrained DBNS with interleaving method.
This algorithm is suitable for number represented in binary-ternary format.



\subheading{Interleaving signed sliding window} \\
\label{sec:signedslide}
The speed leading algorithm for computing single-scalar multiplication is double-and-add with signed sliding window.
The idea of signed sliding window is to precompute a set $S = \{[1]P, [3]P, [5]P, \dots, [2^{\omega}-1]P\}$ where $\omega$ is the window size.
When performing double-and-add, instead of scanning one bit and perform addition,
sliding window keep scanning bits (together perform doubling) until non-zero bit is reach.
From that non-zero bit and within the window width $\omega$, choosing the largest number and perform a single addition.
Then repeat the process of (consecutive) doubling(s) and (single) addition until the last bit is reached.
Note that the main advantage of this method is that it allows many bits (up to $\omega$ bits) to be added at one time.

In this Subsection, we will explain how signed sliding window method can be applied to compute multi-scalar multiplication combining with interleaving method.
The algorithm is very straightforward.  Let use the same example of computing multi-scalar multiplication of $[m]P_1 + [n]P_2$.
Similar to the interleaving method, doubling is mutually performed for both $m$ and $n$ while addition is individually performed for $m$ and $n$.
On top of that, we allow a number to be add to be chosen from precompute set $S_1 = \{[1]P_1, [3]P_1, \dots, [2^{\omega}-1]P_1\}$
and set $S_2 = \{[1]P_2, [3]P_2, \dots, [2^{\omega}-1]P_2\}$.
A selection of which number to be added is similar to the sliding window method that keep sliding and finding a number withing width $\omega$ and is in the precomputation sets.

Let ${\it{FindIndex}}(b,p)$ be a function performing a sliding window of width $\omega$ scanning a bit string $b$ from position $p$ to the right.
This function returns positions $i$ and $j$ closest to $p$ where $b_i$ and $b_j$ are 1 and $(b_i b_{i-1}\ dots b_j)$ is the largest in the precomputation set,
meaning that he difference between $i$ and $j$, namely, $i-j$ must not exceed the window width $\omega$.
The value $i$ and $j$ may be the same if there is only one bit set within the width $\omega$.

Let $R$ keeps the result of $[m]P_1 + [n]P_2$.  The computation starts by initializing $R$ to $0$ and call function $FindIndex$ twice for $m$ and $n$
which it returns $(i_i,j_1)$ and $(i_2,j_2)$ respectively.
Then scan each bit in both $m$ and $n$ from left to right together with performing doubling on $R$.
Let $k$ denote the current position that bits being scanned.
If $k$ is equal to $j_1$, then $m_{i_1} \dots m_{{j_1}+1} m_{j_1}$ is also added to $R$ and $FindIndex(m,j_1-1)$ is called.
Likewise, if $k$ is equal to $j_2$, then $n_{i_2} \dots n_{{j_2}+1} n_{j_2}$ is also added to $R$ and $FindIndex(n,j_2-1)$ is called.
This interleaving signed sliding window algorithm is shown in Algorithm~\ref{interleaveSlidingAlgo}.

%%%%%%%%%%%%%%%%%%%%%%%%%%%%%%%%%%%%%%%%%%%%%%%%%%%%%%%%%%%%%%%%%%%%%%%%%%%%%%%%%%%%%%%%%%%%%%%%%%%%
% Algorithm
%%%%%%%%%%%%%%%%%%%%%%%%%%%%%%%%%%%%%%%%%%%%%%%%%%%%%%%%%%%%%%%%%%%%%%%%%%%%%%%%%%%%%%%%%%%%%%%%%%%%
\begin{algorithm}
\caption{Interleaving signed sliding window}
\begin{algorithmic}
	\Require Scalars $m={(m_{\ell_1},\dots,m_1,m_0)}_2$ and $n={(n_{\ell_2},\dots,n_1,n_0)}_2$,
		points $P_1,P_2$, and sets $S_1=\{1P_1,3P_1,5P_1,\dots,[2^{\omega}-1]P_1\}$,$S_2=\{1P_2,3P_2,5P_2,\dots,[2^{\omega}-1]P_2\}$
	\Ensure Multi-scalar multiplication computing $[m]P_1 + [n]P_2$
	\Statex
	\State Initialize $R \gets 0$
	\State $(i_1,j_1)$ $\gets$ $FindIndex(m,\ell_1)$		\Comment{$FindIndex(b,p)$ returns ($i,j$)}
	\State $(i_2,j_2)$ $\gets$ $FindIndex(n,\ell_2)$		\Comment{where $p {\ge} i {\ge} j, i{-}j {\le} \omega$ and $b_i{,}b_j {=} 1$}
	\If{$j_1 > j_2$} \State $k \gets j_1$
	\Else \State $k \gets j_2$
	\EndIf
	\While {$k > 0$}
		\State $R$ $\gets$ $[2]R$
		\If {$k = j_1$}
			\State Set $u$ $\gets$ ${(m_{i_1},\dots,m_{j_1+1},m_{j_1})}_2$
			\State $R$ $\gets$ $R + [u]P_1$		\Comment{$[u]P_1$ obtained from set $S_1$}
			\State $(i_1,j_1)$ $\gets$ $FindIndex(m,j_1{-}1)$
		\EndIf
		\If {$k = j_2$}
			\State Set $u$ $\gets$ ${(n_{i_2},\dots,n_{j_2+1},n_{j_2})}_2$
			\State $R$ $\gets$ $R + [u]P_2$		\Comment{$[u]P_2$ obtained from set $S_2$}
			\State $(i_2,j_2)$ $\gets$ $FindIndex(n,j_2{-}1)$
		\EndIf
		\State $k$ $\gets$ $k - 1$
	\EndWhile
	\\ \Return $[m]P_1 + [n]P_2$
\end{algorithmic}
\label{interleaveSlidingAlgo}
\end{algorithm}
%%%%%%%%%%%%%%%%%%%%%%%%%%%%%%%%%%%%%%%%%%%%%%%%%%%%%%%%%%%%%%%%%%%%%%%%%%%%%%%%%%%%%%%%%%%%%%%%%%%%

\example{Interleaving signed sliding window} \\
Let $m = m_\ell \dots m_1 m_0 = {100101}_2$ and $n = n_\ell \dots n_1 n_0 = {011001}_2$.
Let $(m_{i_1,j_1})$ and $(n_{i_2,j_2})$ denote $m_{i_1} m_{i_1-1} \dots m_{j}$ and $n_{i_2} n_{i_2-1} \dots n_{j_2}$ respectively.
Let the result of $[m]P_1 + [n]P_2$ be kept in $R$ and let the window width $\omega = 3$.
The computation starts by scanning from $m_\ell$ and find $(m_{i_1,j_1})$ such that $m_{i_1} = m_{j_1} = 1$ and $i_1 - j_1 \le \omega$.
Repeat this with $n$ as well.
In this example, $(m_{i_1,j_1}) = (m_{5,5})$ and $(n_{i_2,j_2}) = (n_{4,3})$.
Because $j_1 > j_2$, $R$ is initialized to $[(m_{5,5})]P_1 = P_1$.
The computation continues by iterating $k$ from $4$ down to $0$.
At each step perform $[2]R$, and if $j_1 = i$ then add $[(m_{i_1,j_1})]P_1$ to $R$, and if $j_2 = i$ then also add $[(n_{i,j_2})]P_2$ to $R$: \\
$k=4; \quad R = 2(P_1) = 2P_1$ \\
$k=3; \quad R = 2(2P_1) + 3P_2 = 4P_1 + 3P_2$ \\
$k=2; \quad R = 2(4P_1 + 3P_2) = 8P_1 + 6P_2$ \\
$k=1; \quad R = 2(8P_1 + 6P_2) = 16P_1 + 12P_2$ \\
$k=0; \quad R = 2(16P_1 + 12P_2) + 5P_1 + P_2 = 37P_1 + 25P_2$ \\


To illustrate the performance of this interleaving signed sliding window algorithm,
we conducted experiments using randomly chosen $10000$ pairs of integers $m$ and $n$.
We used various bit range, namely, 192, 256, 320, 384, 448, and 512.
We also tried using different window widths range from $0$ to $10$.
The performance is measured by the number of multiplication ($N_M$) required to compute $[m]P_1 + [n]P_2$.
To gain the best speed from both doubling and addition, we use mixed coordinate systems,
namely, projective twisted Edwards and extend twisted Edwards with $a=-1$
With this curve choice, a doubling takes $3\mul + 4\sqr$, a regular addition takes $8\mul$, and a mixed addition takes $7\mul$
where $\mul$ and $\sqr$ denote field multiplication and field squaring respectively.
See \cite{EFD} for more details on formulas.
We also use a common assumption that $\sqr \approx 0.8\mul$.
Note that we perform conversion in order to make precomputation points in affine so that a faster mixed addition can be used.
The cost displays in Table~\ref{signedslideTable} does include the cost of precomputation but exclude the cost of conversion.
The notation {$\mathcal{\#P}$} denotes the number of precomputation points.


%%%%%%%%%%%%%%%%%%%%%%%%%%%%%%%%%%%%%%%%%%%%%%%%%%%%%%%%%%%%%%%%%%%%%%%%%%%%%%%%%%%%%%%%%%%%%%%%%%%%
% Table
%%%%%%%%%%%%%%%%%%%%%%%%%%%%%%%%%%%%%%%%%%%%%%%%%%%%%%%%%%%%%%%%%%%%%%%%%%%%%%%%%%%%%%%%%%%%%%%%%%%%
\begin{table}[h]
\centering
\begin{tabular}{|C{0.1\textwidth}| *6{C{0.06\textwidth} C{0.06\textwidth}|} }
%\begin{tabularx}{\textwidth}{|X| *6{X X|}}
\toprule
%\hline
\multirow{2}{*}{width $\omega$}
	&\multicolumn{2}{c|}{192-bit}
		&\multicolumn{2}{c|}{256-bit}
			&\multicolumn{2}{c|}{320-bit}
				&\multicolumn{2}{c|}{384-bit}
					&\multicolumn{2}{c|}{448-bit}
						&\multicolumn{2}{c|}{512-bit} \\
	&\tiny{$N_M$}	&\tiny{$\mathcal{\#P}$}
		&\tiny{$N_M$}	&\tiny{$\mathcal{\#P}$}
			&\tiny{$N_M$}	&\tiny{$\mathcal{\#P}$}
				&\tiny{$N_M$}	&\tiny{$\mathcal{\#P}$}
					&\tiny{$N_M$}	&\tiny{$\mathcal{\#P}$}
						&\tiny{$N_M$}	&\tiny{$\mathcal{\#P}$} \\
\midrule
1 &2103 &2 &2798 &2 &3494 &2 &4189 &2 &4885 &2 &5580 &2 \\
2 &2103 &2 &2798 &2 &3494 &2 &4189 &2 &4885 &2 &5580 &2 \\
3 &1884 &6 &2494 &6 &3104 &6 &3714 &6 &4324 &6 &4935 &6 \\
4 &1838 &10 &2424 &10 &3009 &10 &3595 &10 &4181 &10 &4767 &10 \\
5 &1836 &20 &2391 &21 &2946 &22 &3500 &22 &4056 &22 &4611 &22 \\
6 &1932 &29 &2478 &33 &3016 &36 &3550 &38 &4084 &39 &4618 &40 \\
7 &2049 &36 &2629 &43 &3195 &50 &3752 &55 &4300 &60 &4839 &63 \\
8 &2183 &37 &2786 &47 &3374 &56 &3952 &65 &4520 &72 &5082 &80 \\
9 &2361 &36 &3008 &47 &3631 &57 &4239 &67 &4837 &77 &5425 &86 \\
10 &2531 &34 &3238 &45 &3919 &55 &4571 &65 &5215 &75 &5841 &85 \\

%\hrule
\bottomrule
\multicolumn{13}{c}{}
%\end{tabularx}
\end{tabular}
\caption{Number of multiplications and precomputation points for different window widths to compute multi-scalar multiplication using interleaving signed sliding window}
\label{signedslideTable}
\end{table}
%%%%%%%%%%%%%%%%%%%%%%%%%%%%%%%%%%%%%%%%%%%%%%%%%%%%%%%%%%%%%%%%%%%%%%%%%%%%%%%%%%%%%%%%%%%%%%%%%%%%


We observed that the optimal window width $\omega = 5$.
However, this width requires approximately 20-22 precomputation points on average
which might cause troubles in limited memory environment.
A reasonable choice of width would be $\omega = 4$ which still provides good performance
and only requires 10 precomputation points to be kept.



\subheading{Interleaving DBNS} \\
\label{sec:dbns}
In case of tripling operation is also available in addition to doubling and addition operations,
scalars can be represented using $\{2,3\}$ Double-Base Number System (DBNS)
$$n = \Sigma^{\ell}_{i=1} c_i 2^{a_i}3^{b_i} \text{, where $c_i \in \{-1,1\}$ }.$$
That is, terms in a number expansion consist of power of two and three.

\example{Double-Base Number System} \\
$542788 = 2^83^7 - 2^33^7 + 2^43^3 - 2^13^3 - 2$

The main advantage of DBNS over binary representation is that the number of terms in the expansion is smaller.
However, the main drawback of DBNS is that scalar multiplication cannot be performed using Horner-like method.
Therefore, the cost of doubling is at least $max(a_i)$.  Similarly, the cost of tripling is at least $max(b_i)$.

In \cite{DIM05}, a restrictive DBNS called {\it{double-base chain}} (DBC) was introduced.
Exponents $a_i$ and $b_i$ in the expansion can no long be freely chosen but have to obey the restrictions
$a_\ell \ge a_{\ell-1)} \ge \cdots \ge a_1$ and $b_\ell \ge b_{\ell-1} \ge \cdots \ge b_1$.
These extra conditions on the exponents increase the number of terms in the expansion.
With these restrictions, Horner-like method can be applied to DBC when computing scalar multiplication, e.g.,
$$[841232]P = [3]([3]([3]([2^13^3]([2^63^2]P+P)-P)-P)+P)-P.$$
Unlike DBNS, the cost of doubling and tripling for DBC is guaranteed to be at most $max(a_i)$ and $max(b_i)$.

In \cite{MH09}, M\'eloni and Hasan proposed an algorithm for number represented in DBNS to perform scalar multiplication
in which Horner-like method can still be applicable.
Their idea was to precompute a set containing power of $2$ and apply the Horner-like method only on power of $3$.
The drawback of this algorithm is that it has large amount of precomputation points.

Inspired by that idea, we took a similar approach of using DBNS and try to make it applicable for Horner-like method.
Our approach is, instead of forcing the degree of both exponents $a_i$ and $b_i$ to be in decreasing order,
we force only the degree of exponent $a_i$ to be in decreasing order.
The degree of exponent $b_i$ can be in any order but must not exceed a bound $b^{max}$.
In other words, the restriction becomes
$a_\ell \ge a_{\ell-1} \ge \cdots \ge a_1$ (same as DBC) and $b_1,b_2,\dots,b_\ell \le b^{max}$ (different from DBC).
This way, Horner-like method can now applied to exponent $a_i$.

\example{Apply Horner-like method to DBNS on single exponent} \\
\\
$[983828]P = [2^43^{10}]P - [2^23^4]P + [2^13^9]P + [2^1]P$ \\
$\phantom{[983828]P} = [2]([2^33^{10}]P - [2^13^4]P + [3^9]P + P)$ \\
$\phantom{[983828]P} = [2]([2]([2^23^{10}]P - [3^4]P) + [3^9]P + P)$ \\
\\
Note that $[3^4]P, [3^9]P, [3^{10}]P$ need to be precomputed. \\

In this Subsection, we will explain how to combine the new restricted DBNS with the interleaving method to compute multi-scalar multiplication.
The essential part to compute $[m]P_1 + [n]P_2$ giving a bound $b^{max}$ on the maximum power of 3
is to find a good $\{2,3\}$-DBNS representation for $m$ and $n$.
At each step, we find two approximations, namely, one closest $m$ and the other one closest to $n$.
Then we choose the best one between these two that give the best approximation to both $m$ and $n$ and subtract from them.
Continue to iterate these steps until $m$ and $n$ reach $0$.
This algorithm requires $2 \times b^{max}$ of precomputation points and $2 \times b^{max}$ triplings are used in the precomputation.
The main computation consists of only doublings and additions.
Algorithm~\ref{interleaveDBNSChainAlgo} shows the interleaving DBNS algorithm to obtain a $\{2,3\}$-DBNS.

\example{Generating chain for interleaving DBNS} \\
Let $m = 554627$, $n = 748556$ and $b^{max} = 5$. \\
At each step, let $(a_i,b_i,c_i,d_i)$ where $b_i \le 5$ makes $2^{a_i}3^{b_i}$ the best approximation of $m$ and $n$. \\
$(a_0,b_0) = (18,\phantom{0}1,\phantom{-}0,\phantom{-}1)$;		$\quad n\phantom{m} = \phantom{-}748556			- 2^{18}3^{1}		= -37876$ \\
$(a_1,b_1) = (19,\phantom{0}0,\phantom{-}1,\phantom{-}0)$;		$\quad m\phantom{n} = \phantom{-}554627			- 2^{19}3^{0}		= \phantom{-}30339$ \\
$(a_2,b_2) = (12,\phantom{0}2,\phantom{-}0,-1)$;			$\quad n\phantom{m} = -37876\phantom{0}			+ 2^{12}3^{2}		= -1012$ \\
$(a_3,b_3) = (\phantom{0}7,\phantom{0}5,\phantom{-}1,\phantom{-}0)$; 	$\quad m\phantom{n} = \phantom{-}30339\phantom{0}	- 2^{\phantom{0}7}3^{5} = -765$ \\
$(a_4,b_4) = (10,\phantom{0}0,\phantom{-}0,-1)$;			$\quad n\phantom{m}=  -1012\phantom{00}			+ 2^{10}3^{0}		= \phantom{-}12$ \\
$(a_5,b_5) = (\phantom{0}8,\phantom{0}1,-1,\phantom{-}0)$; 		$\quad m\phantom{n}=  -765\phantom{000}			+ 2^{\phantom{0}8}3^{1} = \phantom{-}3$ \\
$(a_6,b_6) = (\phantom{0}2,\phantom{0}1,\phantom{-}0,\phantom{-}1)$; 	$\quad n\phantom{m}= \phantom{-}12\phantom{0000}	- 2^{\phantom{0}2}3^{1} = \phantom{-}0$ \\
$(a_7,b_7) = (\phantom{0}0,\phantom{0}1,\phantom{-}1,\phantom{-}0)$; 	$\quad m\phantom{n}= \phantom{-}3\phantom{00000}	- 2^{\phantom{0}0}3^{1} = \phantom{-}0$ \\
Therefore, an interleaving DBNS chain: \\
\\
$[554627]P_1 + [748556]P_2 = [2^{18}3^{1}]P_2 + [2^{19}3^{0}]P_1 - [2^{12}3^{2}]P_2 \\
\phantom{[554627]P_1 + [748556]P_2 =}
			+ [2^{7}3^{5}]P_1 - [2^{10}]P_2 - [2^{8}3^{1}]P_1 \\
\phantom{[554627]P_1 + [748556]P_2 =}
			+ [2^{2}3^{1}]P_2 + [2^{0}3^{1}]P_1$. \\
\\
Apply Horner-like method to the power of $2$, then $[554627]P_1 + [748556]P_2$ can be computed as: \\
\\
$[2^2]([2^5]([2]([2^2]([2^2]([2^6]([2]P_1 + [3^1]P_2) - [3^2]P_2) - [3^0]P_2) \\
\phantom{[2^2]([2^5]([2]([2^2]([2^2]([2^6]([2]P_1}
- [3^1]P_1) + [3^5]P_1) + [3^1]P_2) + [3^1]P_1.$ \\
\\
Note: points need to be precomputed are: $[3]P_1, [3^5]P_1, [3]P_2, [3^2]P_2$. \\


%%%%%%%%%%%%%%%%%%%%%%%%%%%%%%%%%%%%%%%%%%%%%%%%%%%%%%%%%%%%%%%%%%%%%%%%%%%%%%%%%%%%%%%%%%%%%%%%%%%%
% Algorithm
%%%%%%%%%%%%%%%%%%%%%%%%%%%%%%%%%%%%%%%%%%%%%%%%%%%%%%%%%%%%%%%%%%%%%%%%%%%%%%%%%%%%%%%%%%%%%%%%%%%%
\begin{algorithm}
\caption{Generating chain for interleaving DBNS}
\begin{algorithmic}
	\Require Scalars $m, n$ and a bound $b^{max}$
	\Ensure	A $\{2,3\}$-DBNS representing $m+n$
	\Statex
	\State Initialize $chain \gets [ ]$
	\While {$m \ne 0$ or $n \ne 0$}
		\State $(a,b,c,d) \gets FindTerm(m,n)$
		\State $m\phantom{n} \gets m - c2^a3^b$
		\State $n\phantom{m} \gets n - d2^a3^b$
		\State $chain \gets chain.append(a,b,c,d)$
	\EndWhile
	\\ \Return $chain$
\end{algorithmic}

\begin{algorithmic}
	\Statex
	\State {\bf{FindTerm($r_1,r_2$):}}
	\State Initialize $min_1 \gets \infty$, $min_2 \gets \infty$
	\For {$0 \le a_i \le a^{max}$ and $0 \le b_i \le b^{max}$}
		\State $d_1 \gets |r_1 - 2^{a_i}3^{b_i}|$
		\If {$d_1 < min_1$}
			\State $min_1 \gets d_1$
			\State $pair_1 \gets (a_i,b_i,sgn(r_1),0)$	\Comment{$sgn(n) = 0$ if $n = 0$; otherwise $sgn(n) = \frac{|n|}{n}$}
		\EndIf
		\State $d_2 \gets |r_2 - 2^{a_i}3^{b_i}|$
		\If {$d_2 < min_2$}
			\State $min_2 \gets d_2$
			\State $pair_2 \gets (a_i,b_i,0,sgn(r_2))$
		\EndIf
	\EndFor
	\If {$min_1^2 + r_2^2 <  r_1^2 + min_2^2$}
		$(a,b,c,d) \gets pair_1$
	\Else
		$(a,b,c,d) \gets pair_2$
	\EndIf
	\\ \Return $(a,b,c,d)$
\end{algorithmic}

\label{interleaveDBNSChainAlgo}
\end{algorithm}
%%%%%%%%%%%%%%%%%%%%%%%%%%%%%%%%%%%%%%%%%%%%%%%%%%%%%%%%%%%%%%%%%%%%%%%%%%%%%%%%%%%%%%%%%%%%%%%%%%%%


%%%%%%%%%%%%%%%%%%%%%%%%%%%%%%%%%%%%%%%%%%%%%%%%%%%%%%%%%%%%%%%%%%%%%%%%%%%%%%%%%%%%%%%%%%%%%%%%%%%%
% Algorithm
%%%%%%%%%%%%%%%%%%%%%%%%%%%%%%%%%%%%%%%%%%%%%%%%%%%%%%%%%%%%%%%%%%%%%%%%%%%%%%%%%%%%%%%%%%%%%%%%%%%%
%\begin{algorithm}
%\caption{Interleaving DBNS}
%\begin{algorithmic}
%	\Require A $chain$ $\{2,3\}$-DBNS Chain representing $m + n$, points $P_1$ and $P_2$
%	\Ensure Multi-scalar multiplication computing $[m]P_1 + [n]P_2$
%	\Statex
%	\State Initialize $R \gets 0$
%	\For {$x$ in $chain$}
%		\State $R \gets [2]R$
%	\EndFor
%	\\ \Return $R$
%\end{algorithmic}
%\label{interleaveDBNSSmultAlgo}
%\end{algorithm}
%%%%%%%%%%%%%%%%%%%%%%%%%%%%%%%%%%%%%%%%%%%%%%%%%%%%%%%%%%%%%%%%%%%%%%%%%%%%%%%%%%%%%%%%%%%%%%%%%%%%


To show the performance of the interleaving DBNS algorithm,
we conducted several experiments and summarized data in Table~\ref{dbnsTable}.
We tested different $b^{max}$ values from $0$ to $15$.
The setting of the experiment  was similar to the one on interleaving signed sliding window ~\ref{sec:signedslide},
namely, we used $10000$ randomly chosen pairs of integers $m$ and $n$ for bit range 192, 256, 320, 384, 448 an 512.
We use the same mixed coordinate systems projective and extended twisted Edwards with $a=-1$.

We also used tripling formulas in this experiment.
Since tripling is used for computing precomputation points which will be added,
outputs of tripling have to be in extend coordinates.
Therefore tripling takes $11\mul + 3\sqr$.  Again, we use a common assumption that $\sqr \approx 0.8\mul$.
See \cite{EFD} for more details on formulas.
Note again that we perform conversion in order to make precomputation point in affine so that a faster mixed addition formulas can be used.
The cost displays in Table~\ref{dbnsTable} does include the cost of precomputation but exclude the cost of conversion.
The notation {$\mathcal{\#P}$} denotes the number of precomputation points.

%%%%%%%%%%%%%%%%%%%%%%%%%%%%%%%%%%%%%%%%%%%%%%%%%%%%%%%%%%%%%%%%%%%%%%%%%%%%%%%%%%%%%%%%%%%%%%%%%%%%
% Table
%%%%%%%%%%%%%%%%%%%%%%%%%%%%%%%%%%%%%%%%%%%%%%%%%%%%%%%%%%%%%%%%%%%%%%%%%%%%%%%%%%%%%%%%%%%%%%%%%%%%
\begin{table}[h]
\centering
\begin{tabular}{|C{0.1\textwidth}| *6{C{0.06\textwidth} C{0.06\textwidth}|} }
%\begin{tabularx}{\textwidth}{|X| *6{X X|}}
\toprule
%\hline
\multirow{2}{*}{$b^{max}$}
	&\multicolumn{2}{c|}{192-bit}
		&\multicolumn{2}{c|}{256-bit}
			&\multicolumn{2}{c|}{320-bit}
				&\multicolumn{2}{c|}{384-bit}
					&\multicolumn{2}{c|}{448-bit}
						&\multicolumn{2}{c|}{512-bit} \\
	&\tiny{$N_M$}	&\tiny{$\mathcal{\#P}$}
		&\tiny{$N_M$}	&\tiny{$\mathcal{\#P}$}
			&\tiny{$N_M$}	&\tiny{$\mathcal{\#P}$}
				&\tiny{$N_M$}	&\tiny{$\mathcal{\#P}$}
					&\tiny{$N_M$}	&\tiny{$\mathcal{\#P}$}
						&\tiny{$N_M$}	&\tiny{$\mathcal{\#P}$} \\
\midrule
0 &2075 &0 &2761 &0 &3464 &0 &4167 &0 &4854 &0 &5534 &0 \\
1 &1917 &2 &2566 &2 &3192 &2 &3845 &2 &4470 &2 &5098 &2 \\
2 &1894 &4 &2507 &4 &3135 &4 &3764 &4 &4379 &4 &4974 &4 \\
3 &1872 &6 &2464 &6 &3076 &6 &3687 &6 &4281 &6 &4868 &6 \\
4 &1836 &8 &2417 &8 &3008 &8 &3586 &8 &4196 &8 &4766 &8 \\
5 &1836 &10 &2410 &10 &3000 &10 &3587 &10 &4170 &10 &4755 &10 \\
6 &1828 &12 &2396 &12 &2983 &12 &3553 &12 &4124 &12 &4708 &12 \\
7 &1837 &14 &2410 &14 &2983 &14 &3544 &14 &4108 &14 &4686 &14 \\
8 &1843 &16 &2407 &16 &2980 &16 &3540 &16 &4105 &16 &4671 &16 \\
9 &1854 &18 &2412 &18 &2976 &18 &3540 &18 &4084 &18 &4656 &18 \\
10 &1859 &20 &2421 &20 &2974 &20 &3530 &20 &4080 &20 &4633 &20 \\
11 &1865 &22 &2422 &22 &2971 &22 &3523 &22 &4069 &22 &4623 &22 \\
12 &1878 &24 &2435 &24 &2973 &24 &3537 &24 &4082 &24 &4629 &24 \\
13 &1883 &26 &2438 &26 &2989 &26 &3543 &26 &4092 &26 &4635 &26 \\
14 &1901 &28 &2444 &28 &3000 &28 &3549 &28 &4092 &28 &4647 &28 \\
15 &1916 &30 &2467 &30 &3018 &30 &3553 &30 &4097 &30 &4653 &30 \\

%\hrule
\bottomrule
\multicolumn{13}{c}{}
%\end{tabularx}
\end{tabular}
\caption{Number of multiplications and precomputation points for different $b^{max}$ to compute double-scalar multiplication using interleaving DBNS}
\label{dbnsTable}
\end{table}
%%%%%%%%%%%%%%%%%%%%%%%%%%%%%%%%%%%%%%%%%%%%%%%%%%%%%%%%%%%%%%%%%%%%%%%%%%%%%%%%%%%%%%%%%%%%%%%%%%%%

The experimental results suggest that setting $b^{max} = 6$ or having $12$ precomputation points gives the best performance for 192- and 256-bit.
The $b^{max}$ value increase as the bit size increase, meaning that more space to keep precomputation points is required as the bit size grows
in order to gain a good performance.



\subheading{Comparison} \\

In Subsection~\ref{sec:signedslide} we introduced interleaving signed sliding window,
and in Subjection~\ref{sec:dbns} we introduced interleaving DBNS.
In each subsection we present the analysis for each algorithm separately.
Therefore, in this Section we will compare results across different algorithms including comparison with previous work

Recall that we use mixed coordinates of projective and extend twisted Edwards with curve parameter $a$ equals $-1$ as a curve choice,
because these coordinate systems provide the best number of multiplication counts.
See \cite{EFD} for more details on formulas.
Normally, a tripling on projective coordinates takes only $9\mul + 3\sqr$,
but in our algorithms the output of triplings will be used for addition on extended coordinates.
Therefore, the tripling needs $2\mul$ more to compute $T$-coordinate which results in the total cost of $11\mul + 3\sqr$.
A doubling takes $3\mul + 4\sqr$, regular addition takes $8\mul$ and mixed addition takes $7\mul$.
We also use a common assumption that $\sqr \approx 0.8\mul$.

To ease a comparison to a previous work \cite{DKS09} which carried out the computation on inverted Edwards coordinates,
we also present our result using operation cost on that coordinate system.
On inverted Edwards coordinates, a tripling requires $9\mul + 4\sqr$, a doubling requires $3\mul + 4\sqr$, a general addition requires $9\mul + 1\sqr$,
and a mixed addition requires $8\mul + 1\sqr$.

Table~\ref{cmpied} displays a comparison between the previous work\cite{DKS09} and our work using cost on inverted Edwards coordinates.
Please note that these costs are calculated using mixed addition, meaning that converting all precomputation points into affine coordinates is necessary.
The costs shown in the table already include the cost to generate precomputation points.
However, these costs do not include the cost to perform conversion.
Note that {\it{Tree-JBT$_5^2$, SignedSlide$_4$, and DBNS$_6$}} denote tree-based joint binary-ternary using all combinations of precompuation points,
signed sliding window $\omega = 4$, and DBNS $b^{max} = 6$ respectively.

The results suggest that interleaving signed sliding window $\omega = 4$ performs the best among these three algorithms.
The interleaving DBNS $b^{max} = 6$ is competitive with tree-base joint binary-ternary.
Even though the DBNS is slightly faster, it requires more precomputation points.

%%%%%%%%%%%%%%%%%%%%%%%%%%%%%%%%%%%%%%%%%%%%%%%%%%%%%%%%%%%%%%%%%%%%%%%%%%%%%%%%%%%%%%%%%%%%%%%%%%%%
% Table
%%%%%%%%%%%%%%%%%%%%%%%%%%%%%%%%%%%%%%%%%%%%%%%%%%%%%%%%%%%%%%%%%%%%%%%%%%%%%%%%%%%%%%%%%%%%%%%%%%%%
\begin{table}[h]
\centering
\begin{tabular}{|L{0.2\textwidth}|C{0.05\textwidth}| *6{C{0.1\textwidth}|} }
%\begin{tabularx}{\textwidth}{|X| *6{X X|}}
\toprule
%\hline
	&Size
		&192-bit
			&256-bit
				&320-bit
					&384-bit
						&448-bit
							&512-bit \\
Method
	&\tiny{$\mathcal{\#P}$}
		&\tiny{$N_M$}
			&\tiny{$N_M$}
				&\tiny{$N_M$}
					&\tiny{$N_M$}
						&\tiny{$N_M$}
							&\tiny{$N_M$} \\
\midrule
% Inverted Edwards %
Tree-JBT$_{5^2}$\cite{DKS09}
		&10 &1890 &2485 &3079 &3677 &4270 &4862 \\
SignedSlide$_4$	&10 &1838 &2424 &3009 &3595 &4181 &4767 \\
DBNS$_6$ [New]	&12 &1888 &2479 &3069 &3668 &4267 &4858 \\

%\hrule
\bottomrule
\multicolumn{8}{c}{}
%\end{tabularx}
\end{tabular}
\caption{Comparison different method measured costs on inverted Edwards coordinates}
\label{cmpied}
\end{table}
%%%%%%%%%%%%%%%%%%%%%%%%%%%%%%%%%%%%%%%%%%%%%%%%%%%%%%%%%%%%%%%%%%%%%%%%%%%%%%%%%%%%%%%%%%%%%%%%%%%%


Table~\ref{cmpted} presents a comparison between the previous work\cite{DKS09} and our work using cost on twisted Edwards coordinates.
Since the previous measures the cost on different coordinate system, based on data provided on their paper we converted their cost into the cost on twisted Edwards coordinates.
In contrast to the cost displays in Table~\ref{cmpied}, the cost displays in this Table is a {\it{complete}} cost.
There is no conversion into affine coordinates involved.  A regular addition is used unless mixed addition is possible.
Note that {\it{Tree-JBT$_5^2$, SignedSlide$_4$, and DBNS$_6$}} denote tree-based joint binary-ternary using all combinations of precompuation points,
signed sliding window $\omega = 4$, and DBNS $b^{max} = 6$ respectively.

The results suggest that interleaving signed sliding window $\omega = 4$ performs the best among at 192-bit.
From 256-bit and above, the interleaving DBNS $b^{max}=6$ performs the best.

%%%%%%%%%%%%%%%%%%%%%%%%%%%%%%%%%%%%%%%%%%%%%%%%%%%%%%%%%%%%%%%%%%%%%%%%%%%%%%%%%%%%%%%%%%%%%%%%%%%%
% Table
%%%%%%%%%%%%%%%%%%%%%%%%%%%%%%%%%%%%%%%%%%%%%%%%%%%%%%%%%%%%%%%%%%%%%%%%%%%%%%%%%%%%%%%%%%%%%%%%%%%%
\begin{table}[h]
\centering
\begin{tabular}{|L{0.2\textwidth}|C{0.05\textwidth}| *6{C{0.1\textwidth}|} }
%\begin{tabularx}{\textwidth}{|X| *6{X X|}}
\toprule
%\hline
	&Size
		&192-bit
			&256-bit
				&320-bit
					&384-bit
						&448-bit
							&512-bit \\
Method
	&\tiny{$\mathcal{\#P}$}
		&\tiny{$N_M$}
			&\tiny{$N_M$}
				&\tiny{$N_M$}
					&\tiny{$N_M$}
						&\tiny{$N_M$}
							&\tiny{$N_M$} \\
\midrule
% Twisted Edwards %
Tree-JBT$_{5^2}$
		&10 &1860 &2445 &3028 &3619 &4204 &4786 \\
SignedSlide$_4$	&10 &1777 &2342 &2913 &3485 &4050 &4622 \\
DBNS$_6$ [New]	&12 &1778 &2329 &2880 \\

%\hrule
\bottomrule
\multicolumn{8}{c}{}
%\end{tabularx}
\end{tabular}
\caption{Comparison different method measured costs on twisted Edwards coordinates}
\label{cmpted}
\end{table}
%%%%%%%%%%%%%%%%%%%%%%%%%%%%%%%%%%%%%%%%%%%%%%%%%%%%%%%%%%%%%%%%%%%%%%%%%%%%%%%%%%%%%%%%%%%%%%%%%%%%










\section{Optimal Algorithm}

In this Section we give a dynamic programming algorithm that can find a DBNS expansion of $m$ and $n$
minimizing the number of additions for the interleaving DBNS discussed in Section \ref{sec:interleaving}.
As will be discussed in Subsection \ref{dynamic}, the computational time obtained from this algorithm is almost equal to the optimal DBNS expansion.
The algorithm will be theoretically analyzed in Subsection \ref{analysis}. The analysis shows that our optimal technique significantly improve techniques in previous works.

\subsection{Dynamic Programming Algorithm}
\label{dynamic}

Let the maximum number of point doublings $I$ and the maximum number of point triplings $J$ are given.
Let 
$$\mathcal{S}_{r, I, J} := \{S_r \subseteq \{-1,1\} \times \{0, 1, \dots, I\} \times \{0, 1, \dots, J\} : v(S_r) = r\},$$
where $v(S_r) = \sum\limits_{(s,i,j) \in S_r} s\cdot 2^i 3^j$.
We will assume that $I,J$ is large enough to have $\mathcal{S}_{r,I,J} \neq \emptyset$.
Our goal is to find $S_r^* \in \mathcal{S}_{r, I, J}$ such that $\left| S_r^* \right| \leq \left| S_r \right|$ for any $S_r \in \mathcal{S}_{r, I, J}$

For each integer $x$, it is known that there exists at most one $T_x \subset \{-1,1\} \times \{0\} \times \{0, 1, \dots, J\}$
such that $\sum\limits_{(s,0,j) \in T_x} s \cdot 3^j = x$.
Let $w$ be a function from $\mathbf{Z}$ to $\mathbf{Z}_{\geq 0} \cup \{\infty\}$ such that $w(x) = |T_x|$ when $T_x$ exists, and $w(x) = \infty$ otherwise. 
For example, when $J \geq 3$, we have $T_{15} = \{(1, 0, 3), (-1, 0, 2), (-1, 0, 1)\}$ and $w(15) = 3$, since $15 = 3^3 - 3^2 - 3^1$. 

In \cite{analysisMethod}, a dynamic programming algorithm for optimizing the number of point additions in fractional window method was proposed.
We will use some of ideas in that method to propose our optimal algorithm, Algorithm~\ref{dynamicAlgorithm}. 

\begin{algorithm}
	\caption{Finding the shortest double-base chain for a given scalar}
	\begin{algorithmic}
		\Require A scalar $r$, a parameter $I, J$
		\Ensure A set $S_r^* \in \mathcal{S}_{r,I,J}$ such that, for any $S_r \in \mathcal{S}_{r,I,J}$, $|S^*_r| \leq |S_r|$. 
		\Statex
		\State $C \leftarrow \{c \in \mathbf{Z} : -\frac{3^{J + 1} - 1}{2} \leq c \leq  \frac{3^{J + 1} - 1}{2}\}$.
		\State For all $x \in \left\{ \lfloor \frac{r}{2^I} \rfloor + c  : c \in C\right\} \cup C$ , calculate $T_x$ and $w(x)$.
		\Comment $T_x$ is a ternary representation of $x$
		 
		\State For all $x \in \left\{ \lfloor \frac{r}{2^I} \rfloor + c  : c \in C\right\}$, $W_{x,I}^* \leftarrow w(x)$ and $S_{x,I}^* \leftarrow T_x$. 
		\Comment formally defined in Subsection~\ref{dynamic},
				
		\For {$i = I - 1$ down to $0$} \Comment and $w(x) = |T_x|$.
			\ForAll { $c \in C$ }
				\State $x \leftarrow \lfloor \frac{r}{2^i} \rfloor + c$.
				\State $d^* \leftarrow \arg \min\limits_{d \in C : d \equiv x \bmod 2} \left[ W^*_{\frac{x - d}{2}, i + 1 } + w(d) \right]$. 
				\State $S_{x,i}^* \leftarrow \left\{(s,i + 1,j) : (s,i,j) \in S^*_{\frac{x - d^*}{2}, i + 1 } \right\} \cup T_{d^*}$.
				\Comment $S_{x,i}^*$ is the smallest set in $\mathcal{S}_{x, I - i, J}$ 
				\State $W_{x,i}^* \leftarrow |S_{x,i}^*|$
				\Comment $W_{x,i}^* = |S_{x,i}^*|$
			\EndFor
		\EndFor
		
		\\ \Return $S_{r,0}^*$
	\end{algorithmic}
	\label{dynamicAlgorithm}
\end{algorithm}

The correctness and optimality of Algorithm~\ref{dynamicAlgorithm} is proved in the following theorem. 

\begin{theorem}
The set $S^*_{r,0}$ found by Algorithm~\ref{dynamicAlgorithm} is the smallest set in $\mathcal{S}_{x,I,J}$
\end{theorem}

\begin{proof}
We will prove this theorem by showing that all $S^*_{x,i}$ found by Algorithm~\ref{dynamicAlgorithm} is the smallest set in $\mathcal{S}_{x, I - i, J}$.

We know that all members of $\mathcal{S}_{r, I, J}$ can be written in the form of $\sum\limits_{i = 0}^I \left(\sum\limits_{j = 0}^J s(i,j) \cdot 3^j\right) 2^i$
for some $s(i,j) \in \{0, \pm 1\}$. Let $d(i) = \sum\limits_{j = 0}^J s(i,j) \cdot 3^j$.
We have $\sum\limits_{i = 0}^I \left(\sum\limits_{j = 0}^J s(i,j) \cdot 3^j\right) 2^i = \sum\limits_{i = 0}^I d(i) 2^i$
and  $-\frac{3^{J + 1} - 1}{2} \leq d(i) \leq \frac{3^{J + 1} - 1}{2}$.
Instead of finding the optimal DBNS, we find $d(0), d(1), \dots, d(I)$ that optimize the number of point additions, which is $\sum_{i = 0}^I w(d(i))$.

The set $C$ represents all possible values of $d(i)$.
At Line 2, we calculate $T_x$ and $w(x)$ for each member of $C$.
It is easy to see that $\mathcal{S}_{x, 0, J} = \{T_x\}$ if $T_x$ exists
and $\mathcal{S}_{x, 0, J} = \emptyset$ if $T_x$ does not exist.
Hence, $S_{x, I}^* = T_x$ assigned in Line 3 is the smallest set in $\mathcal{S}_{x, 0, J}$  and $W_{x, I}^* = |T_x|$.

For $i < I$, we will select $S^*_{x,i}$ from the set $S^*_{x, I - i, J}$.
This is equivalent to writing $x$ in the form $\sum_{i = 0}^{I - i} d(i) 2^i$
for some $d(0), \dots, d(I - i)$ or $2 \left( \sum_{i = 0}^{I - i - 1} d(i + 1) 2^i \right) + d(0)$.
When $d(0)$ is fixed, we know that the best choice for $d(1), \dots, d(I - i)$ is the sequence that is corresponding to the smallest set in
$\mathcal{S}_{\frac{x - d(0)}{2}, I - i - 1, J}$, $S^*_{\frac{x - d(0)}{2}, i + 1}$.
Therefore, our task is to find a value $d(0) \in C$ that minimizes $\left|S^*_{\frac{x - d(0)}{2}, i + 1}\right| + \left|T_{d(0)}\right|$ which is $d^*$ in our algorithm.
The set $S^*_{x,i}$ in Line 8 is corresponding to best $d(0), \dots, d(I - i)$ obtained from the idea in this paragraph.

For all $d(0) \in C$, we have $\lfloor \frac{x}{2} \rfloor - \frac{3^{J + 1} - 1}{2} \leq \frac{x - d(0)}{2} \leq \lfloor \frac{x}{2} \rfloor + \frac{3^{J + 1} - 1}{2}$.
Hence, all $S^*_{\frac{x - d(0)}{2}, i + 1}$ referred at Step $i$ are computed at Step $i + 1$. \qed
\end{proof}  

\begin{example} Find $S^*_{15} \in \mathcal{S}_{15, 2, 1}$ such that $|S^*_r| \leq |S_r|$ for any $S_r \in \mathcal{S}_{15, 2, 1}$

Since $J = 1$,
$C = \{0, \pm 1, \pm 2, \pm 3, \pm 4\}$,
$T_0 = \emptyset$,
$T_1 = \{(1,0,0)\}$,
$T_2 =\{(-1,0,0), (1,0,1)\}$,
$T_3 = \{(1,0,1)\}$,
$T_4 = \{(1,0,0),(1,0,1)\}$,
$T_{-x} = \{(-d, i, j) : (d, i, j) \in T_x\}$,
$w(0) = 0$,
$w(1) = w(3) = w(-1) = w(-3) = 1$, and
$w(2) = w(4) = w(-2) = w(-4) = 2$.
We also calculate $T_x$ and $w(x)$ for each $x \in \{ \lfloor \frac{r}{2^I} \rfloor + c : c \in C \} = \{ -1, 0, 1, \dots, 7 \}$ in Line 2.
We know from the calculation that $T_5, T_6$ and $T_7$ do not exist, and $w(5) = w(6) = w(7) = \infty$.

In the first iteration of the loop in Algorithm~\ref{dynamicAlgorithm}, $i = I - 1 = 1$.
We have $\lfloor \frac{r}{2^1} \rfloor = 7$.
When $c = 0$,
$x = 7$, 
$W^*_{(7 - (-3)) / 2,2} + w(-3) = w(5) + w(-3) = \infty$,
$W^*_{(7 - (-1)) / 2,2} + w(-1) = w(4) + w(1) = 3$,
$W^*_{(7 - 1) / 2,2} + w(1) = w(3) + w(1) = 2$, and
$W^*_{(7-3)/2, 2} + w(3) = w(2) + w(3) = 3$.
By that, the value of $d^*$ is $1$,
$S^*_{7, 1} \leftarrow \left\{(s,i+1,j) : (s,i,j) \in S^*_{(7-1) / 2, 2}\right\} \cup T_1 = \{ (1,1,1) \} \cup \{ (1,0,0)\} = \{ (1,1,1), (1,0,0) \}$.
In the same iteration, we also calculate $S^*_{x,1}$ and $W^*_{x,1}$ for all $x \in \{ 3, 4, \dots, 11 \}$.

In the second iteration, we calculate $S^*_{x,0}$ and $W^*_{x,0}$ for all $x \in \{ 11, \dots, 19 \}$.
Let us show the calculation for $S^*_{15,0}$.
Since we have $W^*_{(15 - (-3)) / 2,1} + w(-3) = 2 + 1 = 3$,
$W^*_{(15 - (-1)) / 2,1} + w(-1) = 2 + 1 = 3$, $W^*_{(15 - 1) / 2,1} + w(1) = 2 + 1 = 3$, and
$W^*_{(15-3)/2, 2} + w(3) = 1 + 1 = 2$.
The value of $d^*$ is equal to $3$.
$S^*_{15,0} = \left\{(s,i+1,j) : (s,i,j) \in S^*_{(15-1) / 2, 1}\right\} \cup T_3 = \{(1,2,1)\} \cup \{(1,0,1)\} = \{(1,2,1), (1,0,1)\}$. \qed
\end{example}

To optimize the number of additions for the calculation of $mP_1 + nP_2$ using the interleaving DBNS in the previous section,
we can compute $S_m^*$ and $S_n^*$ independently by Algorithm~\ref{dynamicAlgorithm}.
Then, we replace the $\{2,3\}$-DBNS obtained from Algorithm~\ref{interleaveDBNSChainAlgo} by
$$chain^* := \left\{ (a,b,0,d) : (d, a, b) \in S_m^* \right\} \cup \left\{ (a,b,c,0) : (c, a, b) \in S_n^* \right\}$$
in the calculation of the multi-scalar multiplication.

The smallest set in $\mathcal{S}_{r, I, J}$, $S^*_r$ might not lead us to the fastest interleaving DBNS.
For each $S_r \in \mathcal{S}_{r, I, J}$, we need $|S_r|$ point additions,
$M_i(S^*_r) := \max \{i : (i,j) \in S_r\}$ point doublings, and $M_j(S^*_r) := \max \{i : (i,j) \in S_r\}$ point triplings.
There might be a set $S'_r \in \mathcal{S}_{r, I, J}$ of which $|S'_r| = |S^*_r|$, but $M_i(S'_r) < M_i(S^*_r)$ or $M_j(S'_r) < M_j(S^*_r)$.
However, it can be seen that, for any $S_r, S_r' \in \mathcal{S}_{r, I, J}$, $|M_i(S_r) - M_i(S'_r)| \leq 2 \cdot J$ and $|M_j(S_r) - M_j(S'_r)| \leq J$.
As $J$ is small, we can say that a computation using $S^*_r$ is almost the fastest among all $S_r$ in $\mathcal{S}_{r,I,J}$.

Although Algorithm~\ref{dynamicAlgorithm} potentially leads to a faster multi-scalar multiplication than Algorithm~\ref{interleaveDBNSChainAlgo},
the computational time of the former is clearly larger than the latter.
When the value of $J$ is equal to $7$, the size of $C$ is $6781$.
We have to calculate the value of $S^*_{x,i}$ for different $6781I$ values.
However, in most cryptographic protocols, scalars $m$ and $n$ are usually private- or public-keys.
We perform several multi-scalar multiplications on embedded system using the same $m$ and $n$ after those numbers are generated on a personal computer. 
Since it is reasonable to spend time and memory for finding the best DBNS in the computation environment,
we believe that Algorithm~\ref{dynamicAlgorithm} can be useful for those practical situations.
Besides, we found that the running time of Algorithm~\ref{dynamicAlgorithm} is not larger than $3$ seconds,
for all $r \leq 2^{256}$, $I \leq 256$, and $J \leq 6$.
By an argument in~\cite{experiment}, we strongly believe that $3$ seconds is not too long in many applications.

\subsection{Analysis}
\label{analysis}

Beside the experimental results, one way to evaluate a multi-scalar multiplication that is commonly used in literature is \textit{average Hamming density}.
The average joint Hamming density of a method $\mathcal{A}$ can be described as follows:
$$AJD_\mathcal{A} := \lim\limits_{l \rightarrow \infty} \sum_{m = 0}^{2^l - 1} \sum_{n = 0}^{2^l - 1} \frac{\text{\#Additions for } mP_1 + nP_1}{l \cdot 2^{2l}}.$$
That is the average number of additions per bits when the number of bits is significantly large. 

In~\cite{analysisMethod}, the general framework for finding the average Hamming density is proposed.
By that framework, our density compared to the fractional window method~\cite{fractional} and enlarged digit set~\cite{enlarged4} is shown in Table~\ref{optimalTable}.
Although the results when the number of precomputation points is larger than $8$ is still on-going due to the limitation of our computation environment,
the table shows that our algorithm significantly improve the previous works when $\#P \geq 4$.
Our average Hamming density for $\#P = 8$ is even smaller than the value for $\#P = 10$ in both methods.

\begin{table}[h]
\centering
\begin{tabular}{|L{0.2\textwidth}|*6{C{0.12\textwidth}|} }
%\begin{tabularx}{\textwidth}{|X| *6{X X|}}
\toprule
%\hline
Method
	&$\#P = 0$
		&$\#P = 2$
			&$\#P = 4$
				&$\#P = 6$
					&$\#P = 8$
						&$\#P = 10$\\
\midrule
% Twisted Edwards %
Frac. Windows \cite{fractional}
		& $0.667$ & $0.500$& $0.486$ & $0.455$ & $0.435$ \\ %& $0.417$ \\
Enlarged Set \cite{enlarged4}	& - &$0.500$ & - & - & - \\ %& $0.358$ \\
Optimal DBNS [New]	&$0.667$ &$0.500$ & $0.432$ &$0.384$ &$0.354$ \\ % & On-going\\

%\hrule
\bottomrule
\multicolumn{6}{c}{}
%\end{tabularx}
\end{tabular}
\caption{Comparison different methods measured average Hamming density}
\label{optimalTable}
\end{table}

Beside the analysis on the average Hamming density, our experimental results also prove that the dynamic programming algorithm performs better than any methods in literature.
The average number of point multiplications are $1865$, $2439$, $3012$, $3585$, $4157$, and $4731$
when the number of bits in the inverted Edwards coordinates are $192$, $256$, $320$, $384$, $448$, and $512$.
That improves the best method in literature, the tree-based JBT~\cite{DKS09}, by $1.35 - 2.78\%$, and improves our greedy algorithm by up to $2.17\%$.
For the twisted Edwards coordinates, our average number of point multiplications are $1835$, $2397$, $2955$, $3512$, $4069$, and $4626$.
The results are better than the best method in literature, the interleaving method, by $0.17 - 3.04\%$, and are better than our greedy algorithm by up to $1.09\%$.

\section{Comparison}

add text here

\subheading{This is subheading}
add more text

\section{Conclusion}

add conclusion here



\clearpage
\bibliographystyle{plain}
\bibliography{collection}

\end{document}
